\chapter{Visual Studio Code: Einstieg und Tipps}
\label{sec:vscode}

Visual Studio Code (VS Code) ist ein moderner, kostenloser Editor, der besonders für Programmieranfänger:innen viele Vorteile bietet. 
In diesem Abschnitt zeigen wir, wie ihr VS Code installiert, sinnvoll einrichtet und für den Vorkurs nutzt.

\section{Installation}
\begin{enumerate}
    \item Ladet VS Code von \url{https://code.visualstudio.com/Download} herunter und installiert es.
    \item Öffnet VS Code nach der Installation.
\end{enumerate}

\section{Empfohlene Erweiterungen}
\begin{itemize}
    \item \textbf{C/C++} (Microsoft): Syntax-Highlighting, Autovervollständigung und Debugging für C++.
    \item \textbf{CodeLLDB} oder \textbf{C/C++ Extension Pack}: Für erweitertes Debugging.
    \item \textbf{Remote - WSL} (nur Windows): Für die Verbindung zu WSL (Windows Subsystem for Linux).
    \item \textbf{Better Comments}, \textbf{Bracket Pair Colorizer}, \textbf{GitLens}: Für mehr Übersicht und Komfort.
\end{itemize}

\section{VS Code mit WSL verbinden (nur Windows)}
\begin{enumerate}
    \item Installiert das Windows-Subsystem für Linux (siehe Kapitel \ref{sec:windows}).
    \item Installiert die Erweiterung \textbf{Remote - WSL} in VS Code.
    \item Öffnet die WSL-Konsole und gebt \texttt{code .} ein, um das aktuelle Verzeichnis in VS Code zu öffnen.
    \item VS Code erkennt automatisch, dass ihr im WSL arbeitet.
\end{enumerate}

\section{Debugger einrichten}
\begin{enumerate}
    \item Öffnet die Datei, die ihr debuggen wollt.
    \item Klickt links auf das Symbol für „Run and Debug“ (\texttt{Play}-Button mit Käfer).
    \item Wählt „C++ (GDB/LLDB)“ oder „C++ (Windows)“ aus.
    \item Erstellt ggf. eine \texttt{launch.json} (VS Code bietet eine automatische Konfiguration an).
    \item Setzt Breakpoints durch Klick auf die Zeilennummer.
    \item Startet das Debugging mit F5.
\end{enumerate}

\section{Nützliche Shortcuts}
Eine vollständige Übersicht findet ihr im Anhang \ref{sec:cheatsheet-shortcuts}. Hier die wichtigsten:
\begin{itemize}
    \item \texttt{Strg + P}: Datei schnell öffnen
    \item \texttt{Strg + Shift + P}: Befehlspalette öffnen
    \item \texttt{Strg + \textasciigrave}: Terminal öffnen/schließen
    \item \texttt{F5}: Debugging starten
    \item \texttt{F9}: Breakpoint setzen/entfernen
\end{itemize}

\section{Weitere Tipps}
\begin{itemize}
    \item Ihr könnt mehrere Terminals gleichzeitig öffnen (z.B. Bash und PowerShell).
    \item Die integrierte Git-Unterstützung hilft beim Versionsmanagement.
    \item Mit \texttt{settings.json} könnt ihr VS Code individuell anpassen.
\end{itemize}

\section{Fehlerbehebung}
\begin{itemize}
    \item Prüft, ob alle benötigten Erweiterungen installiert sind.
    \item Bei Problemen mit WSL: VS Code und WSL neu starten.
    \item Compiler-Fehler erscheinen im „Problems“-Tab unten.
\end{itemize}