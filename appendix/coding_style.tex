\lesson{Coding style}
\label{sec:codingstyle}
Wir haben mittlerweile hinreichend viele verschiedene Verschachtelungen im
Quellcode kennen gelernt, dass es sich lohnt, ein paar Worte über coding styĺe
zu sprechen.

Schon wenn ihr euch während dieses Kurses an einen von uns wendet und um Hilfe
bittet, ergibt sich das Problem der Lesbarkeit von Quellcode. Um euch zu
helfen, sollte man möglichst mit einem Blick erfassen können, was euer Code
tut, wie er strukturiert ist, welche Variable was bedeutet. Um dies zu
unterstützen, gibt es mehrere Dinge, auf die man achten kann.

\begin{description}
    \item[Einrückung]
          Wie euch vermutlich aufgefallen ist, sind an verschiedenen Stellen im
          Code einzelne Zeilen ein wenig eingerückt. Dies ist vermutlich das
          wichtigste Werkzeug, welches zur Verfügung steht, um die Lesbarkeit von
          Code zu unterstützen (auch, wenn es nicht nötig ist, um formal korrekte
          Programme zu schreiben). Die einzelnen Einheiten des Kontrollflusss
          werden dadurch visuell voneinander abgegrenzt, was es einfacher macht,
          den Programmverlauf zu verfolgen.

          Wie genau eingerückt werden sollte, darüber scheiden sich die Geister.
          Man kann mit mehreren Leerzeichen oder durch Tabulatoren einrücken.
          Empfehlenswert ist auf jeden Fall, mehrere gleichförmige „Ebenen“ zu
          haben (z.B. 4, 8, 12, \dots\ Leerzeichen zu Beginn der Zeile). Eine
          Faustregel für gut lesbare Einrückung ist, immer wenn man eine
          geschweifte Klammer öffnet, eine Ebene tiefer einzurücken und immer,
          wenn man eine geschweifte Klammer schließt, wieder eine Ebene zurück zu
          nehmen.
    \item[Klammern]
          Aus der Schule kennt ihr das Prinzip „Punkt- vor Strichrechnung“. Dies
          ist eine Regel, die so genannte \emph{Präzedenz} ausdrückt, also die
          Reihenfolge, in der Operatoren ausgewertet werden. Punkt vor Strich ist
          allerdings nicht aussreichend, um vollständig zu beschreiben, wie sich
          Operatoren in Gruppen verhalten. Schaut euch z.B. den Ausdruck
          \texttt{3 * 2 / 3} an. Da der Computer Ganzzahldivision benutzt, kommen
          hier unterschiedliche Ergebniss raus, je nachdem, ob zunächst das
          \texttt{*} oder das \texttt{/} ausgewertet wird. Im ersten Fall
          erhalten wir \texttt{6 / 3 == 2}, wie wir erwarten würden. Im zweiten
          Fall wird aber abgerundet, sodass wir \texttt{3 * 0 == 0} erhalten.

          Um solche und ähnliche Uneindeutigkeiten zu vermeiden, bietet es sich
          an, Klammerung zu verwenden. Selbst wenn wir im obigen Fall
          \emph{wissen} in welcher Reihenfolge die Operatoren ausgewertet werden,
          jemand der unseren Code liest, weiß das vielleicht nicht. Einfach von
          vornherein die gewollte Reihenfolge der Auswertung zu klammern,
          verhindert Verwirrung bei uns über das Verhalten des Computers, als
          auch bei Menschen, die später wissen wollen, was wir meinten.

          Ihr könnt übrigens nicht nur einzeilige Kommentare erstellen, die mit \cppinline{//} beginnen, sondern auch mehrzeilige, und zwar so: \cppinline{/* Dies ist ein ganz langer mehrzeiliger Kommentar.  */} . Alles zwischen den Slashes und Sternchen ist dann ein Kommentar und wird vom Computer ignoriert. Dies könnt ihr als kleinen Trick verwenden, um euren Code zu debuggen, ohne ständig alles neu zu schreiben. Ihr könnt stattdessen einfach den nicht benötigten Code auskommentieren und wenn ihr ihn wieder verwenden wollt, die Kommentarzeichen am Anfang und Ende wieder entfernen.
    \item[Kommentare]
          Wir haben schon in mehreren Quellcodedateien Kommentare verwendet, um
          einzelne Dinge zu erklären. Insgesamt bietet es sich an, dies selbst
          ebenfalls zu tun, um den Programmfluss der Leserin von Quellcode klar zu
          machen. Das heißt nicht, dass man jede Anweisung noch einmal mit einem
          Kommantar versehen sollte, der sie ausführlich erklärt, aber an
          wichtigen Punkten können einem kurze Kommentare das Leben enorm
          vereinfachen. Und ihr dürft nicht vergessen, dass ihr euch vielleicht
          selbst in ein oder zwei Jahren noch einmal euren eigenen Quellcode
          anschauen müsst und ihr werdet wirklich überrascht sein, wie wenig ihr
          von dem Zeug, welches ihr selbst geschrieben habt, verstehen werdet.
    \item [Benennungen]
          Wenn ihr eure Variablen -- und später auch eure Funktionen und Klassen -- präzise benennt, dann vereinfacht ihr das Lesen eures Codes extrem. Durch Bezeichnungen, die für sich sprechen, könnt ihr euch außerdem Kommentare etwas ersparen, weil die Variablennamen dann schon viel erklären. Es ist zum Beispiel ungeschickt, seine Variablen wie in der Mathematik üblich einfach nur mit einzelnen Buchstaben zu benennen, statt \cppinline{int a = 42;} sollte man lieber \cppinline{int alter = 42;} verwenden, da die Leserin direkt weiß, dass in dieser Variablen das Alter gespeichert wird. Zu dieser coding style Richtlinie gibt es jedoch auch eine Ausnahme: Bei Zählervariablen, die einfach nur die Anzahl der Schleifeniterationen hochzählen, verwendet man meist einzelne Buchstaben, wie \cppinline{i} oder \cppinline{n}. Das ist schön kurz und praktisch, jedoch muss man etwas aufpassen, denn man kann schnell mit diesen Indices durcheinander kommen -- genau so wie in der Mathematik.
    \item[Leerzeichen und -zeilen]
          Weniger wichtig als die ersten vier Punkte können trotzdem gezielte
          Leerzeichen (z.B. zwischen Operatoren und Operanden in arithmetischen
          Ausdrücken) die Lesbarkeit enorm erhöhen. Gerade in arithmetischen
          Ausdrücken ist es eine gute Angewohnheit.
          Ebenso sind Leerzeilen zwischen logischen Abschnitten sehr hilfreich. Zu Beginn eines Abschnittes kann man dann noch einen kurzen Kommentar hinzufügen, was in dem Abschnitt passiert und schon fällt das Lesen des Codes deutlich leichter.
\end{description}

Es gibt sicher noch viele Regeln, über die ihr im Laufe eures Lebens stolpern
werdet, wenn ihr euch entschließen solltet, regelmäßig zu programmieren. Häufig
werdet ihr euch darüber ärgern, manchmal zu recht. Aber versucht im Zweifel
einen halbwegs sauberen Stil auch als euren eigenen Verbündeten zu sehen, denn
ob es nun vergessene Klammern, Semikolons oder versteckte Fehler in der
Operatorpräzedenz sind, ein sauberer Stil kann euch bei allen enorm helfen, sie
aufzuspüren. Auch wenn es coding style Richtlinien für verschiedene Programmiersprachen gibt, die größtenteils relativ ähnlich sind, gewöhnt man sich meist einen eigenen Stil mit der Zeit an. Es ist allerdings wichtig, früh auf guten Stil zu achten, denn wenn man erst einmal damit anfängt, unübersichtlichen Code zu schreiben, gewöhnt man sich diese Unart an und das will schließlich niemand.

\begin{praxis}
    \begin{enumerate}
        \item Eine weit verbreitete einfache Aufgabe, die in Bewerbungsgesprächen
              auf Stellen als Programmiererin häufig gestellt wird, ist
              \emph{FizzBuzz}. In \texttt{fizzbuzz.cpp} ist eine möglich Lösung für
              diese Aufgabe gegeben. Könnt ihr (nur mittels des Quellcodes) sagen,
              was das Programm tut?
        \item Nutzt die oben gegebenen Faustregeln, um den Quellcode lesbarer zu
              machen. Ihr müsst nicht alles bis aufs Wort befolgen, macht einfach so
              lange weiter, bis ihr findet, man kann hinreichend schnell verstehen,
              was hier passieren soll.
    \end{enumerate}

    \inputcpp{fizzbuzz.cpp}
\end{praxis}

\begin{spiel}
    \begin{enumerate}
        \item Entfernt in eurem veränderten Quellcode eine geschweifte Klammer
              eurer Wahl. Lasst eure Sitznachbarin über den Quellcode schauen und die
              fehlende Klammer finden.
    \end{enumerate}
\end{spiel}

\textbf{Quiz 12}\\
\textit{Was gehört zu gutem Coding style?}
\begin{enumerate}[label=\alph*)]
    \item Sinnvolle Kommentare
    \item Möglichst keine Leerzeilen lassen
    \item Variablennamen möglichst kurz wählen
    \item Einrückungen vornehmen
\end{enumerate}
