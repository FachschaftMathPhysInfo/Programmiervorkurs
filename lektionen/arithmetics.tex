\lesson{Arithmetik}

Wir haben in der vergangenen Lektion Variablen vom Typ \texttt{std::string}
kennengelernt. Zeichenketten zu speichern ist schon einmal ein guter Anfang,
aber wir wollen auch rechnen können, wir brauchen also mehr Typen für
Variablen.

\Cpp unterstützt eine Unmenge an Datentypen und hat auch die Möglichkeit,
eigene zu definieren. Wir wollen uns hier nur mit den wichtigsten beschäftigen.

Fangen wir mit dem wohl meist genutzten Datentyp an: Einem \texttt{int}, oder
\texttt{integer}. Dieser speichert eine ganze Zahl (mit bestimmten Grenzen, an
die wir aber erst einmal nicht stossen werden, von daher ignorieren wir sie
erst einmal frech). Mit \texttt{int}s können wir rechnen, das funktioniert in
\Cpp mit ganz normalen Rechenausdrücken, wie wir sie aus der Schule kennen,
plus den bereits angetroffenen Zuweisungen:

\inputcpp{arith1.cpp}

Wichtig ist hier, zu beachten, dass wir dem Computer ein in Reihenfolge
abgearbeitetes Programm geben, keine Reihe von Aussagen. Das bedeutet in diesem
konkreten Fall, dass wir z.B. nicht die Aussage treffen „\texttt{a} ist gleich
7“, sondern dass wir sagen „lasse zuerst \texttt{a} den Wert 7 haben. Lasse
dann \texttt{b} den Wert 19 haben. Lasse dann \texttt{c} den Wert haben, der
heraus kommt, wenn man den Wert von \texttt{b} vom Wert von \texttt{a}
abzieht“. Besonders deutlich wird dieser Unterschied bei einem Beispiel wie
diesem:

\inputcpp{arith2.cpp}

\begin{praxis}
	\begin{enumerate}
		\item Was gibt dieses Programm aus? Überlegt es euch zuerst und kompiliert
		      es dann, um es auszuprobieren.
	\end{enumerate}

	Obwohl \texttt{a = a + 19} mathematisch überhaupt keinen Sinn ergibt, ist doch
	klar, was passiert, wenn man sich den Quellcode eben nicht als Reihe von
	Aussagen, sondern als Folge von \emph{Anweisungen} vorstellt. Das
	Gleichheitszeichen bedeutet dann nicht, dass beide Seiten gleich sein sollen,
	sondern dass der Wert auf der linken Seite den Wert auf der rechten Seite
	annehmen soll.

	Wie wir in diesem Beispiel ausserdem sehen, können wir nicht nur Strings
	ausgeben, sondern auch Zahlen. \texttt{std::cout} gibt sie in einer Form aus,
	in der wir etwas damit anfangen können. Genauso können wir auch über
	\texttt{std::cin} Zahlen vom Benutzer entgegen nehmen:

	\inputcpp{arith3.cpp}

	Langsam aber sicher tasten wir uns an nützliche Programme heran!

	\begin{enumerate}[resume]
		\item Schreibt ein Programm, welches von der Nutzerin zwei ganze Zahlen
		      entgegen nimmt und anschließend Summe, Differenz, Produkt und Quotient
		      ausspuckt.
		\item Was fällt auf, wenn ihr z.B. 19 und 7 eingebt?
		\item Findet heraus (Google ist euer Freund), wie man in \Cpp Division mit
		      Rest durchführt und gebt diese zusätzlich zu den bisherigen Operationen
		      mit aus\footnote{Falls ihr nicht weiterkommt, hilft euch vielleicht das
			      Stichwort „modulo“ oder „modulo-operator“ weiter.}.
		\item Was passiert, wenn ihr als zweite Zahl eine 0 eingebt?
	\end{enumerate}
\end{praxis}

\begin{spiel}
	\begin{enumerate}
		\item Findet heraus, was die größte positive (und was die kleinste
		      negative) Zahl ist, die ihr in einem \texttt{int} speichern könnt.
		      Faulpelze nutzen Google, Lernbegierige versuchen sie experimentell zu
		      ermitteln. Was passiert, wenn ihr eine größere Zahl eingebt?
		\item Wir arbeiten bisher nur mit \texttt{int}s für ganze Zahlen. Wenn wir
		      mit gebrochenen Zahlen rechnen wollen brauchen wir den Datentyp
		      \texttt{double}. Schreibt euer Mini Rechenprogramm so um, dass es statt
		      \texttt{int}s nur noch \texttt{double} benutzt und probiert es aus.
		      Achtet darauf, dass es Dezimalpunkte und Dezimalkommata gibt, wenn ihr
		      überraschende Ergebnisse erhaltet.
	\end{enumerate}
\end{spiel}

\textbf{Quiz 7}\\
\textit{Was passiert, wenn ihr \texttt{int} verwendet, aber eine Kommazahl eingebt?}
\begin{enumerate}[label=\alph*)]
	\item Alles hinter dem Komma wird abgeschnitten
	\item Es tritt ein Fehler auf
	\item Das Programm kompiliert nicht
	\item statt \texttt{int} wird automatisch \texttt{double} genommen
\end{enumerate}

