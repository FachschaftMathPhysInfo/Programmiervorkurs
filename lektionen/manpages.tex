\lesson{Manpages}

Wir machen eine kurze Pause vom \Cpp und schauen uns in der Zwischenzeit
\emph{man pages} an. Wie wir bereits fest gestellt haben, kann man diese
benutzen, um sich mehr Informationen über Befehle anzeigen zu lassen. Wir
wollen uns jetzt genauer anschauen, wie man all die Informationen in einer man
page am Besten konsumiert.

Wir schauen uns das am Beispiel der Manpage \texttt{man cp} an (\texttt{cp} ist
der Befehl zum Kopieren von Dateien).

\begin{praxis}
    \begin{enumerate}
        \item Öffnet eine Konsole und gebt \texttt{man cp} ein.
    \end{enumerate}
\end{praxis}

    Die man page besteht aus mehreren \emph{Sections}. Welche sections genau es
    gibt, hängt von der man page ab, aber meistens gibt es mindestens die folgenden
    sections:
    \begin{description}
        \item[\texttt{NAME}]
              Gibt euch den Namen des Befehls und eine Einzeilige Beschreibung an
        \item[\texttt{SYNOPSIS}]
              Gibt euch die generelle Benutzung des Befehls an. In diesem Fall gibt
              es drei mögliche Formen. Allen gemein ist, dass man zunächst
              \texttt{cp} eingibt, darauf folgen Optionen. Wie der Rest interpretiert
              wird, hängt dann vom Rest ab. Werden zwei weitere Parameter angegeben,
              wird der erste als Quelle, der zweite als Ziel interpretiert (erste
              Form). Werden mehr Parameter angegeben, wird das letzte als
              Verzeichnis, in das man alle anderen kopieren will interpretiert
              (zweite Form). In der dritten Form (wenn \texttt{-t} angegeben wird)
              wird hingegen der \emph{erste} Parameter als das Zielverzeichnis
              interpretiert, in das alle anderen Dateien kopiert wird.

              Es gibt eine Vielzahl von Konventionen für diesen Bereich, eckige
              Klammern bedeuten z.B. dass dieser Teil auch weggelassen werden darf,
              drei Punkte bedeuten, dass hier mehrere solche Dinge stehen können.

              Dieser Bereich ist der, der am Interessantesten für euch ist, wenn ihr
              „einfach schnell wissen wollt, wie es funktioniert“.
        \item[\texttt{DESCRIPTION}]
              Hier wird ausführlicher beschrieben, was der Befehl tut. Hier werden
              auch alle möglichen Optionen beschrieben, die wir dem Befehl bei
              \texttt{[OPTION]...} mitgeben können. Die wichtigen Informationen
              stehen meistens irgendwo in diesem Bereich.
        \item[\texttt{AUTHOR}, \texttt{REPORTING BUGS}, \dots]
              Hier stehen weitere Hintergrundinformationen, die meistens eher für
              Entwicklerinnen interessant sind.
        \item[\texttt{SEE ALSO}]
              Auch eine wichtige section für euch: Wenn ihr die gewünschte
              Information nicht gefunden habt, oder ihr nicht den richtigen Befehl
              gefunden habt, stehen hier manchmal verwandte Befehle oder Quellen
              weiterer Informationen.
    \end{description}

    Man pages sind häufig sehr umfangreich und enthalten viel mehr Informationen,
    als ihr euch gerade wünscht. Es ist nicht immer einfach, die gerade relevanten
    Informationen heraus zu filtern und es gibt nichts frustrierenderes, als einen
    Befehl gerade dringend zu brauchen, aber nicht zu kennen und sich erst durch
    eine lange man page lesen zu müssen.

    Dennoch ist es eine sehr hilfreiche Fähigkeit, zu wissen, wie man man pages
    liest und sich einfach in einem ruhigen Moment mal durch die ein oder andere
    man page durch zu lesen. Häufig lernt man dabei neue Dinge, manchmal macht es
    einem das Leben irgendwann sehr viel leichter, sie zu wissen.

    Habt von daher Geduld, wenn euch eine wirsche Linux-Expertin auf die Frage, wie
    ihr unter Linux euren Laptop in den Ruhemodus versetzt ein schnelles „man
    pm-suspend“ antwortet. Mit ein bisschen Übung wird euch das tatsächlich
    hinreichend schnell zur richtigen Lösung verhelfen.


Und wenn ihr mal wirklich keine Zeit habt, die ganze page zu lesen, könnt ihr mit \texttt{/} auch nach Begriffen innerhalb der page suchen: Zum Beispiel \texttt{/close}

\begin{praxis}
      \begin{enumerate}\addtocounter{enumi}{1}
      \item Öffnet die man page von \texttt{ls}. Findet die Optionen fürs Lange
                              Listenformat (long listing format), zum Sortieren nach Dateigröße
                              und um auch versteckte Dateien (unter Linux sind das alle, die mit
                              \texttt{.} anfangen) anzuzeigen und probiert sie aus.
      \item Was ist der Unterschied zwischen \texttt{ls -a} und \texttt{ls -A}?
            Probiert beides aus. Das ist auf den ersten Blick nicht so leicht zu sehen
                              Fragt uns im einfach wenn ihr es nicht findest.
      \item Nutzt \texttt{cp} um eine Datei zu kopieren. Sucht euch dafür
            irgendeine \texttt{.cpp}-Datei aus dem Vorkurs-Programm und kopiert sie
            in euer Homeverzeichnis (ihr könnt dafür eine Tilde (\texttt{\~})
            benutzen).
      \end{enumerate}
\end{praxis}

\begin{spiel}
    \begin{enumerate}
        \item Wie über so gut wie jeden Befehl gibt es auch über \texttt{man} eine
              manpage. Schaut euch mal \texttt{man man} an.
        \item Befehle, die für euch im späteren Leben interessant sein könnten sind
              z.B. \texttt{ls}, \texttt{cp}, \texttt{mkdir}, \texttt{grep}, \texttt{cat},
              \texttt{echo}, \texttt{mv}, \dots. Ihr könnt ja schon einmal in ein
              oder zwei dieser manpages hinein schauen, und ein oder zwei Befehle
              ausprobieren. Aber ihr müsst das jetzt auf keinen fall alles im Kopf
              behalten.
    \end{enumerate}
\end{spiel}

\textbf{Quiz 6}\\
\textit{Was findet man alles in einer Manpage?}
\begin{enumerate}[label=\alph*)]
    \item Nützliche Informationen
    \item viel Text
    \item wie man einen Befehl verwendet
    \item warum der Befehl erfunden wurde
\end{enumerate}
