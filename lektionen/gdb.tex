\lesson{Der Debugger}

\textbf{Fehlerklassen}

Es ist wichtig, früh zu verstehen, dass es verschiedene Klassen von Fehlern in
einem \Cpp Programm gibt, die sich alle zu unterschiedlichen Zeitpunkten
auswirken. Die Hauptsächliche Klasse von Fehlern, die wir bisher betrachtet
haben, sind \emph{Compilerfehler}. Sie treten -- wie der Name nahe legt -- zur
Compilezeit auf, also wenn ihr euer Programm kompilieren wollt. Meistens
handelt es sich hier um relativ einfach erkennbare Fehler in der Syntax (wie
zum Beispiel ein vergessenes Semikolon, oder eine vergessene geschweifte
Klammer), um fehlende header (wie die \verb|\#include <...>| heißen) oder um
undefinierte Variablen.

Eine andere, besonders fiese Klasse von Fehlern haben wir in der Lektion des Kontrollflusses gelernt.
Wenn wir nämlich durch eine Variable teilen, und in
dieser Variable erst beim Programmlauf (zur \emph{Laufzeit}) eine 0 steht, so
tritt eine so genannte \emph{floating point exception} auf. Der Compiler hat
hier keine Chance, diesen Fehler zu erkennen - er weiß ja nicht, was der
Benutzer später hier eingibt! Da diese Klasse von Fehlern zur Laufzeit auftritt
heißen sie Laufzeitfehler. Und sie sind immer ein Zeichen von fundamentalen
Fehlern im Programm. Sie sind also die am schwersten aufzutreibenden Fehler, da
es keine automatischen Tools gibt, die uns bei ihrer Suche helfen.

\textbf{gdb}

Wir haben zwar bereits bei einem so offensichtlichen Fehler bereits gelernt,
wie wir diesen beheben können, dies ist allerdings nicht immer so einfach.
Wir lernen nun ein wichtiges Tool kennen, um Laufzeitfehler auch bei komplexeren Programmen
aufzuspüren, sodass wir wissen, wo wir mit der Lösung
ansetzen müssen: Den \emph{GNU debugger}, oder kurz gdb.

Der Debugger ist eine Möglichkeit, unser Programm in einer besonderen Umgebung
laufen zu lassen, die es uns erlaubt es jederzeit anzuhalten. So kann man den Inhalt von
Variablen untersuchen oder auch Anweisung für Anweisung das Programm durchgehen.

Damit wir das können, braucht der Compiler ein paar zusätzliche
Informationen über den Quellcode, die normalerweise verloren gehen. Wir müssen
dem Compiler dafür ein paar zusätzliche Optionen mitgeben:
\begin{minted}{bash}
g++ -O0 -g3 -o debugger debugger.cpp
\end{minted}
(Beachtet, dass im ersten Parameter erst ein großer Buchstabe o, dann eine 0 stehen)

\newpage

\begin{praxis}

    Wir möchten uns nun den Ablauf des Programms \texttt{debugger.cpp} anschauen.
    \inputcpp{debugger.cpp}

    \begin{enumerate}
        \item Kompiliert das Programm mit den neuen Optionen für den debugger. Ihr
              könnt es dann mittels \verb|gdb ./debugger| im gdb starten. Ihr solltet
              nun ein wenig Text ausgegeben bekommen und einen anderen prompt
              (\texttt{(gdb)}). Ihr könnt den debugger jederzeit wieder verlassen,
              indem ihr \texttt{quit} eingebt (falls ihr gefragt werdet, ob ihr euch
              sicher seid, gebt \texttt{y} ein und drückt enter)
        \item Zu allererst müssen wir einen so genannten \emph{breakpoint} setzen,
              das ist ein Punkt im Programmablauf, an dem es stoppen soll, damit wir
              entscheiden können, was wir tun wollen. \texttt{main} ist für die
              meisten unserer Programme eine sichere Wahl:
              \mint{text}|break main|
              Dann können wir das Programm mit \texttt{run} starten. Wir sollten die
              erste Anweisung unseres Programmes angezeigt bekommen.
        \item Der Debugger wird euch jetzt immer sagen, welches die nächste
              Anweisung ist, die er ausführen möchte. Mit \texttt{next} könnt ihr sie
              ausführen lassen, mit \texttt{print a} könnt ihr euch den Inhalt von
              \texttt{a} zu diesem Zeitpunkt anschauen, mit \texttt{print b} den von
              \texttt{b} und so weiter. Geht das Programm Schritt für Schritt durch
              und lasst euch die Werte von \texttt{a}, \texttt{b} und \texttt{c} in
              jedem Schritt ausgeben. Wenn der debugger euch sagt, dass euer Programm
              beendet wurde, gebt \texttt{quit} ein und beendet ihn.
    \end{enumerate}
\end{praxis}

\begin{spiel}
\begin{enumerate}
        \item Im folgenden Programm \texttt{factorial.cpp} haben sich zwei Fehler eingeschlichen. Versucht diese mit \texttt{gdb} zu finden und zu beheben.
        
        \inputcpp{factorial.cpp}
        
        \item Ihr habt nun schon einige Programme kennen gelernt. Kompiliert sie
              für den Debugger neu und untersucht sie genauso wie obiges Programm,
              solange ihr Lust habt.
\end{enumerate}
\end{spiel}

\textbf{Quiz 11}\\
\textit{Was kann der Debugger?}
\begin{enumerate}[label=\alph*)]
    \item Beim Fehler finden helfen
    \item Fehler automatisch korrigieren
    \item Das Programm Befehl für Befehl durchgehen
    \item Dateien, die eigentlich einen Fehler werfen kompilieren
\end{enumerate}
