\lesson{Fehlermeldungen und häufige Fehler}

Wenn ihr in den vergangen Lektionen ein bisschen herumprobiert habt, wird es
euch sicher das ein oder andere mal passiert sein, dass euch der Compiler statt
eines funktionierenden Programms eine Riesenmenge Fehlermeldungen ausgespuckt
hat und ihr einen Schreck bekamt und schon dachtet, ihr hättet alles kaputt
gemacht.

\texttt{g++} ist leider bei Fehlermeldungen immer sehr ausführlich und gibt
euch lieber viel zu viel, als viel zu wenig aus. Das kann im ersten Blick ein
bisschen überwältigend wirken, aber wenn man einmal gelernt hat, wie die
Fehlermeldungen am Besten zu lesen sind, ist das alles gar nicht mehr so
schlimm.

Wir schieben deswegen eine Lektion über häufige Fehlerquellen ein und wie man
Fehlermeldungen von \texttt{g++} liest, um möglichst schnell die Ursache des
Fehlers zu finden.

Nehmen wir z.B. mal folgendes Programm:

\inputcpp{fehler1.cpp}

Wenn wir versuchen, dieses zu kompilieren, gibt uns \texttt{g++} folgendes aus:

\begin{textcode*}{label=g++ -o fehler1 fehler1.cpp}
    fehler1.cpp: In function 'int main()':
    fehler1.cpp:2:5: error: 'cout' is not a member of 'std'
    fehler1.cpp:2:35: error: 'endl' is not a member of 'std'
\end{textcode*}

Wenn wir diese Fehlermeldung verstehen wollen, fangen wir immer ganz oben an,
egal wie viel Text uns der Compiler ausspucken mag. In diesem Fall sagt uns die
erste Zeile, in welcher Datei (\texttt{fehler1.cpp}) der Fehler aufgetreten ist
und in welcher Funktion (\texttt{int main()}). Die beiden Zeilen
danach sind sogar noch spezifischer: Sie enthalten zu Beginn den Dateinamen,
dann einen Doppelpunkt, gefolgt von einer Zeilennummer, gefolgt von einer
Spaltennummer. Das gibt euch ganz genau die Stelle an, an der der Compiler
etwas an eurem Code zu bemängeln hat. In diesen Fall ist, was der Compiler
bemängelt, dass \texttt{cout} bzw. \texttt{endl} nicht in \texttt{std} sind.
Was genau \texttt{std} bedeutet muss uns nicht interessieren, aber der Rest
sagt uns (mit ein bisschen Erfahrung) dass wir die Definition von \texttt{cout}
und \texttt{endl} nicht haben - was nicht weiter verwunderlich ist, denn diese
beiden Dinge werden in der Datei \texttt{iostream} definiert, die wir früher
immer includiert haben.

Damit wissen wir jetzt auch (endlich) was das \mint{c++}|#include <iostream>|
zu bedeuten hatte. Offenbar brauchen wir das, wenn wir Konsolen input und
output machen wollen, da es die Definitionen von \texttt{cout}, \texttt{cin},
\texttt{endl} und ähnlichem enthält.

Der nächste sehr häufig vorkommende Fehler ist subtiler:

\inputcpp{fehler2.cpp}

Wenn wir versuchen, dies zu kompilieren, bekommen wir vom Compiler
entgegengespuckt:

\begin{textcode*}{label=g++ -o fehler2 fehler2.cpp}
    fehler2.cpp: In function 'int main()':
    fehler2.cpp:5:1: error: expected ';' before '}' token
\end{textcode*}

Wiederum sagt uns die erste Zeile, in welcher Datei und Funktion der Fehler
aufgetreten ist. Die zweite Zeile sagt uns wo, nämlich in Zeile 5, direkt am
Anfang. Die Beschwerde des Compilers ist, dass er ein Semikolon erwartet hat,
aber eine geschlossene geschweifte Klammer gefunden hat. Der Grund dafür ist,
dass in \Cpp erwartet wird, dass jede Anweisung mit einem Semikolon abgeschlossen
wird.  Wenn ihr euch die bisherigen Quellcodedateien anschaut, werdet ihr
feststellen, dass hinter den allermeisten Zeilen ein solches Semikolon steht.
Hier fehlt es allerdings nach der Ausgabe in Zeile 4. Sobald wir es hinzufügen,
beschwert sich der Compiler nicht mehr.

Hier zeigt sich eine ein bisschen verwirrende Angewohnheit von Fehlermeldungen
von \Cpp: Obwohl der Compiler behauptete, der Fehler läge in Zeile 5, lag er in
Wahrheit bereits in Zeile 4. Hier müsst ihr dem dummen Compiler ein wenig
nachsichtig sein - er kann es einfach nicht besser wissen. Wenn ihr also mal in
der richtigen Zeilennummer nachschlagt, aber nicht wisst, wo dort der Fehler
sein sollte, schaut vielleicht mal ein oder zwei Zeilen darüber, vielleicht
wusste der Compiler es einfach nicht besser.

\textbf{Praxis:}
\begin{enumerate}
    \item Versucht, folgende Dateien zu kompilieren und schaut euch die
        Fehlermeldung an. In welcher Zeile, in welcher Spalte liegt der Fehler?
        Was gibt euch der Compiler als Fehlermeldung aus?
    \item Versucht, die aufgetretenen Fehler zu korrigieren. Bekommt ihr es
        hin, dass der Compiler sich nicht mehr beschwert und das Programm
        korrekt arbeitet (schaut euch ggf. die bisher gezeigten Quellcodes an)?
\end{enumerate}

\inputcpp{fehler3.cpp}
\inputcpp{fehler4.cpp}

\begin{spiel}
\begin{enumerate}
    \item Das folgende Programm enthält mehrere Fehler. Bekommt ihr trotzdem
        raus, welche das sind und könnt ihr sie beheben (Tipp: „c++ math“ zu
        \href{https://lmgtfy.com/?q=c\%2B\%2B+math}{googlen} kann euch hier vielleicht weiter bringen)?
    \item Wenn ihr in den vergangenen Lektionen ein bisschen gespielt habt und
        vereinzelt versucht habt, Dinge zu löschen, werden euch viele
        Fehlermeldungen begegnet sein, versucht, diese zu lesen und
        interpretiert, was euch der compiler hier sagen will.
\end{enumerate}
\end{spiel}

\inputcpp{fehler5.cpp}

\textbf{Quiz 4}\\
\textit{Was hiervon sind Fehler, die dazu führen, dass eine Datei nicht kompiliert werden kann?}
\begin{enumerate}[label=\alph*)]
    \item Semikolon vergessen
    \item include vergessen
    \item return vergessen
    \item Einrückung falsch
\end{enumerate}
