\lesson{Der Kontrollfluss}

Nachdem wir ein bisschen etwas über den Debugger verstanden haben (wir werden
ihn noch häufiger benutzen), können wir uns nun wieder unserem Problem mit der
Division durch 0 zuwenden.

\inputcpp{arith4.cpp}

Wenn wir dieses Programm kompilieren und als zweite Zahl eine 0 eingeben,
werden wir auf der Konsole ausgegeben bekommen:
\begin{minted}{text}
Gebe eine Zahl ein: 5
Gebe noch eine Zahl ein: 0
Floating point exception
\end{minted}
(Gegebenenfalls ist die letzte Zeile bei euch auch in einer anderen Sprache)

Wir können das Programm auch einmal im debugger ausführen und werden wenig
überraschend feststellen, dass die Anweisung, an der diese floating point
exception auftritt die ist, in der die Division steht.

Wenn wir diesen Fehler beheben wollen, haben wir eigentlich nur zwei
Möglichkeiten: Die erste ist, die Schuld auf die Benutzerin zu schieben, warum
versucht sie auch, eine 0 einzugeben? Ich hoffe, ihr stimmt zu, dass das nicht
sehr freundlich wäre. Stellt euch vor, jedes mal, wenn ihr in einem Programm
einen Wert eingebt, auf den das Programm nicht vorbereitet ist, würde es direkt
abstürzen. Das fändet ihr vermutlich nicht so gut, es sollte doch zumindest mal
eine Fehlermeldung ausgeben und die Nutzerin informieren, dass sie was falsch
gemacht hat.

Und das ist der zweite Weg, den wir jetzt einschlagen wollen. Unser Programm
sollte am Besten, nachdem es die Eingabe von der Benutzerin entgegen genommen
hat, einfach überprüfen, ob die Division erlaubt ist oder nicht. Sollte die
Nutzerin eine 0 eingegeben haben, sollte es auf den Fehler hinweisen und sich
beenden, sonst sollte es den Quotienten ausgeben. Diese Abhängigkeit des
Verhaltens eines Programms von den Eingaben, bezeichnen wir als
\emph{Kontrollfluss}, man kann das mit einem Diagramm verdeutlichen:

\begin{center}
    \begin{tikzpicture}[auto, node distance=3cm,>=latex']
        \tikzstyle{block} = [draw, fill=blue!20, rectangle, minimum height=3em, minimum width=6em]

        \node [block] (start) {Input};
        \node [block, right of=start] (if) { $a=0$? };
        \node [block, right of=if, node distance=4cm] (fehler) { Gib Fehler aus };
        \node [block, below of=fehler,node distance =  2cm] (quotient) { Gib Quotient aus };
        \node [block, right of=fehler, node distance = 3.5cm] (ende) { Ende };

        \draw [->] (start) -- node {} (if);
        \draw [->] (if) -- node {\texttt{ja}} (fehler);
        \draw [->] (if.south) |- node [above, near end] {\texttt{nein}} (quotient);
        \draw [->] (quotient) -| node {} (ende);
        \draw [->] (fehler) -- node {} (ende);
    \end{tikzpicture}
\end{center}

Die einfachste Möglichkeit, den Kontrollfluss zu ändern, besteht in so
genannten „bedingten Anweisungen“:
\inputcpp{if.cpp}

In den Zeilen 12 bis 20 sehen wir, wie eine solche Bedingte Anweisung in \Cpp
aussieht. Wir erkennen relativ direkt unser Diagramm hier wieder: In Zeile 12
steht der „$b=0$?“ Block, in den Zeilen 13 bis 17 steht der „Gib Fehler aus“
Block und in Zeile 19 der „Gib den Quotienten aus“ Block. Die Blöcke lassen sich auch gut anhand der geschweiften Klammern und der Einrückung erkennen. Wir empfehlen euch, solche logischen Blöcke einzurücken, um die Lesbarkeit des Codes zu verbessern. Eine ausführlichere Erklärung zum coding style gibt es in Lektion \ref{sec:codingstyle}.

Beachtet allerding die doppelten Gleichheitszeichen in Zeile 12. \Cpp hat
getrennte Operatoren für Vergleiche und Zuweisungen - Doppelte
Gleichheitszeichen bedeuten Vergleich („sind diese beiden gleich?“), ein
einfaches Gleichheitszeichen bedeutet Zuweisung („mache diese beiden gleich!“).

\textbf{Praxis:}
\begin{enumerate}
    \item Kompiliert \texttt{if.cpp} für den debugger und lasst das Programm im
          gdb laufen. Geht Schritt für Schritt durch das Programm, mit
          verschiedenen Eingaben (wenn ihr am Ende des Programms angekommen seid,
          könnt ihr es mit einem erneuten „run“ neu starten)
    \item Nutzt Google, um herauszufinden, welche anderen Vergleichsoperatoren
          es in \Cpp noch gibt. Versucht, das Programm so zu verändern, dass es
          auf Ungleichheit testet, statt auf Gleichheit (sich sonst aber genauso
          verhält).
    \item Wie würdet ihr testen, ob zwei Zahlen durch einander teilbar sind
          (Tipp: Ihr kennt bereits die Division mit Rest in \Cpp (modulo))?
          Schreibt ein Programm, welches zwei Zahlen von der Nutzerin entgegen
          nimmt und ausgibt, ob die zweite Zahl die erste teilt.
\end{enumerate}

\textbf{Spiel:}
\begin{enumerate}
    \item Testet mit verschiedenen Eingaben, was passiert, wenn ihr in
          \texttt{if.cpp} statt zwei Gleichheitszeichen nur eines benutzt.
          Benutzt den debugger, um euch den Inhalt von \texttt{b} vor und nach
          dem Test anzuschauen.
    \item Schreibt ein Programm, welches die Benutzerin fragt, wie sie heißt.
          Gibt sie euren eigenen Namen ein, soll das Programm begeistert über die
          Namensgleichheit sein, sonst sie einfach begrüßen.
\end{enumerate}
