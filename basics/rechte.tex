\lesson{Dateirechte}

Wir machen mal wieder eine kurze Pause von \Cpp um euch ein weiteres wichtiges
Konzept der Linux-Welt nahe zu bringen: Dateirechte.

Unter Windows seid ihr es wahrscheinlich gewohnt, dass der Dateiname festlegt,
wie mit der Datei umgegangen wird -- eine \texttt{.doc} wird in Word geöffnet,
eine \texttt{.zip} in einem installierten Packprogramm, eine \texttt{.bmp}
vermutlich in Windows Paint und eine \texttt{.exe} wird ausgeführt.

Das Konzept der Dateierweiterung hat es auch in die Linuxwelt geschafft, ist
hier aber deutlich weniger wichtig. Insbesondere gibt es keine Dateierweiterung
\texttt{.exe}. Stattdessen hat jede Datei einen bestimmten Modus. Eine Datei
kann ausführbar sein, oder nicht. Sie kann lesbar sein, oder nicht. Sie kann
schreibbar sein, oder nicht. Nicht nur das, jede Datei gehört auch einer
bestimmten Nutzerin und einer bestimmten Nutzerinnengruppe und Ausführbarkeit,
Lesbarkeit oder Schreibbarkeit ist getrennt eingestellt für die Besitzerin der
Datei, der Gruppe, der die Datei gehört und für alle anderen. Eine Datei kann
also z.B. lesbar sein, für alle Nutzerinnen, aber nur eine bestimmte Gruppe von
Nutzerinnen darf sie ausführen und nur eine einzige Nutzerin sie bearbeiten. All
dies wird in neun so gennanten \emph{Permission bits} festgehalten (ein
\emph{Bit} ist die kleinste Einheit an Information, es kodiert genau „ja“ und
„nein“, oder „null“ und „eins“, oder „ein“ und „aus“).

Ihr könnt euch die Besitzerin, die Gruppe, und die permission bits einer Datei
mithilfe von \texttt{ls -l} anschauen. Der output von \texttt{ls -l} ist in
mehreren Spalten angeordnet:
\begin{enumerate}
    \item In der ersten Spalte stehen die Dateiberechtigungen in Form eines 10
          Zeichen langen Strings. Jedes Zeichen steht dabei für ein permission
          bit kann dabei entweder ein \texttt{-}, oder ein Buchstabe sein, wobei
          \texttt{-} bedeutet, dass das entsprechende Bit nicht gesetzt ist. Die
          Bits bedeuten (von links nach rechts gelesen)
          \begin{itemize}
              \item \texttt{\underline{d}irectory}
              \item \texttt{\underline{r}eadable} für die Eigentümerin
              \item \texttt{\underline{w}ritable} für die Eigentümerin
              \item \texttt{e\underline{x}ecutable} für die Eigentümerin
              \item \texttt{\underline{r}eadable} für die Gruppe
              \item \texttt{\underline{w}ritable} für die Gruppe
              \item \texttt{e\underline{x}ecutable} für die Gruppe
              \item \texttt{\underline{r}eadable} für alle Nutzerinnen
              \item \texttt{\underline{w}ritable} für alle Nutzerinnen
              \item \texttt{e\underline{x}ecutable} für alle Nutzerinnen
          \end{itemize}
    \item Nummer an hardlinks (das braucht euch nicht sonderlich interessieren)
    \item Nutzername der Eigentümerin
    \item Gruppe, der die Datei gehört
    \item Dateigröße
    \item Datum der letzten Änderung
    \item Dateiname
\end{enumerate}

Wenn ihr die Berechtigungen von Dateien ändern wollt, könnt ihr dazu
\texttt{chmod} benutzen (wenn ihr wissen wollt, wie man es benutzt: \texttt{man
    chmod}), dazu muss sie euch aber gehören. Wenn ihr die Eigentümerin einer Datei
ändern wollt, könnt ihr dazu \texttt{chown} nutzen -- dazu müsst ihr aus
Sicherheitsgründen allerdings Administratorin sein.

\begin{praxis}
    \begin{enumerate}
        \item Geht in ein Verzeichnis, in dem eine \texttt{.cpp}-Datei liegt und
              kompiliert sie. Macht ein \texttt{ls -l} und vergleicht die Rechte der
              \texttt{.cpp}-Datei mit der kompilierten Datei.
        \item In der Datei \texttt{/etc/shadow} stehen in verschlüsselter Form
              gespeichert die Kennwörter aller Benutzerinnen auf dem System. Macht ein
              \texttt{ls -l /etc/shadow} und schaut euch die Dateirechte an. Welche
              Bits sind gesetzt?
    \end{enumerate}
\end{praxis}

\begin{spiel}
    \begin{enumerate}
        \item Versucht, \texttt{/etc/shadow} in einem Editor zu öffnen.
        \item Legt (z.B. mit dem Texteditor) eine Datei (Es geht nicht um
              Kompilierung, also muss das keine \texttt{.cpp}-Datei sein. Gebt der
              Datei am Besten die Erweiterung \texttt{.txt}) in Eurem Homeverzeichnis
              an und macht sie dann mit \texttt{chmod a+w} world-writable
              (\texttt{a+w} heißt „füge das Recht Schreibbarkeit für alle Nutzerinnen
              hinzu“).  Lasst eure Sitznachbarin die Datei an ihrem Rechner öffnen
              (ihr könnt mittels \texttt{pwd} herausfinden, in welchem Ordner sie
              suchen muss) und euch eine Nachricht hinein schreiben. Schaut nach
              (indem ihr die Datei neu öffnet) ob ihr die Nachricht lesen könnt.
    \end{enumerate}
\end{spiel}