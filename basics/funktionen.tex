\lesson{Funktionen}

Aus der Mathematik kennt ihr bereits Funktionen, wie zum Beispiel $f(x) = x^2$.
Eine wichtige Idee dahinter ist es, einfach $f(2.5)$ zu schreiben, wenn man eigentlich $2.5^2$ meint.
In diesem simplen Beispiel bilden wir einfach eine reelle Zahl ab und erhalten als Ergebnis wieder eine reelle Zahl.
So ähnlich findet sich das Konzept von Funktionen auch in Programmiersprachen wie \Cpp wieder.

Eine Funktion in \Cpp besteht aus zwei Teilen: der \emph{Signatur} und dem \emph{Funktionsrumpf}.
Die Kombination von Parametertypen und Rückgabetyp bildet die Signatur einer Funktion.
Parameter sind Werte, die der Funktion übergeben werden, zum Beispiel das $x$ in $f(x)$.
Für eine Funktion \cppinline{my_func}, die  $x^n$ berechnen soll, könnte eine Signatur so aussehen:
\[
	\smashoperator{\mathop{\underbrace{\text{\cppinline{double}}}}_{\text{Rückgabetyp}}}\quad
	\smashoperator{\mathop{\overbrace{\text{\cppinline{my_func}}}}^{\text{Name}}}
	(\smashoperator{\mathop{\underbrace{\text{\cppinline{double x}}}}_{\text{Paramter 1}}},\
	\smashoperator{\mathop{\overbrace{\text{\cppinline{int n}}}^{\text{Paramter 2}}}})
\]
%sorry etwas hässlich

\begin{itemize}
	\item Rückgabetyp: Bestimmt welchen Datentyp die Rückgabe der Funktion hat
	\item Name: Ein frei wählbarer Name für die Funktion 
	\item Parameter: Besteht aus dem Datentype des Parameters und dem beliebig wählbaren Parameternamen. \\Mehrere Parameter können angegeben werden, indem man diese durch Kommata voneinander trennt.
\end{itemize}

In diesem Fall ist also \texttt{double} der Rückgabewert, \cppinline{my_func} der Name, \cppinline{x} ein Parameter mit dem Typ \texttt{double} und \cppinline{n} ein Paramter mit dem Typ \texttt{int}.
Damit können dann Werte an die Funktion in der Form \cppinline{my_func(1.41, 2)} übergeben werden.

An dieser Stelle ist der Unterschied zwischen Rückgabe und Ausgabe wichtig: Eine Ausgabe (gekennzeichnet durch \cppinline{std::cout}) gibt Informationen auf dem Bildschirm für die Nutzerin aus, eine Rückgabe (gekennzeichnet durch \texttt{return}) gibt hingegen ein bestimmtes Ergebnis an einen anderen Teil des Programms zurück, damit dieser dort in einer Variable gespeichert oder direkt weiter verarbeitet werden kann.

Der Funktionsrumpf beinhaltet den Code, der beim Funktionsaufruf tatsächlich ausgeführt wird.
Dieser wird wie in einer Schleife von \mintinline{c++}|{| und \mintinline{c++}|}| umschlossen.
Innerhalb dieser Klammern kann dann beliebiger Code ausgeführt werden, wie auch in der \texttt{main}-Funktion.
Dabei kann auf die Parameter einfach mit dem in der Signatur definierten Namen zugegriffen werden.
Also in unserem Beispiel mit \cppinline{x} und \cppinline{n}.
Vor dem Ende des Funktionsrumpfes muss eine Rückgabe mit \texttt{return} ausgeführt werden.\\
\newpage
Eine Funktion, die $ x^n $ berechnet und ein paar mal aufgerufen wird, könnte dann wie folgt aussehen:

\inputcpp{beispielfunktion.cpp}

Unsere Funktion wird in diesem Beispiel 4-mal aufgerufen.
Das erste mal werden konkrete Werte als Parameter übergeben.\\
Beim zweiten Aufruf übergeben wir eine Variable, anstelle eines konkreten Werts.\\
Und dann? In Zeile 14 rufen wir \cppinline{my_func} mit dem Ergebnis eines weiteren Funktionsaufrufs auf, ohne dieses zuerst in einer Variablen zwischenzuspeichern.
Dabei kann man sich vorstellen, dass der Funktionsaufruf nach dem die Funktion ausgeführt wurde durch den Rückgabewert ersetzt wird.
Dies könnte für die Funktion \cppinline{my_func} folgendermaßen aussehen:

\begin{center}
	\cppinline{f(5.0 + f(3.0, 2), 3)} $\mapsto$ \cppinline{f(5.0 + 9.0, 3)} $\mapsto$ \cppinline{f(14.0, 3)} $\mapsto$ \cppinline{2744}
\end{center}




Funktionen werden beispielsweise benötigt, wenn bestimmte Programmteile häufiger mit verschiedenen Parametern ausgeführt werden sollen.
Die Collatz-Vermutung\footnote{\url{https://de.wikipedia.org/wiki/Collatz-Vermutung}} besagt für die Folge:
\[
	x_n =
	\begin{cases}
		\frac{x_{n-1}}{2}   & x_{n-1} \text{ ist gerade}   \\
		3 \cdot x_{n-1} + 1 & x_{n-1} \text{ ist ungerade}
	\end{cases}
\]
dass jeder Startwert $x_1$ aus den natürlichen Zahlen nach endlich vielen Schritten bei der $1$ angelangt.
Zum Beispiel für den Startwert $x_1 = 42$:

\[
	42 \mapsto 21 \mapsto 64 \mapsto 32 \mapsto 16 \mapsto 8 \mapsto 4 \mapsto 2 \mapsto 1 \mapsto 4 \mapsto 2 \mapsto 1 \mapsto \ldots
\]

Wenn nun die Frage aufkommt was die nächsten Folgenglieder von verschiedenen Zahlen sind, wäre ein möglicher Lösungsweg eine Funktion zu schreiben, die der Nutzerin die nächste Zahl in dieser Folge zurückgibt.

\inputcpp{funktion.cpp}

\begin{praxis}\footnote{In dieser Lektion gibt es ein paar mehr Aufgaben als in anderen Lektionen, lasst euch davon nicht entmutigen!}
	\begin{enumerate}
		\item Verändert das Programm in \texttt{funktion.cpp} so, dass es nicht die einzelnen Zahlen \texttt{x1}, \texttt{x2} und \texttt{x3}, sondern die Summe dieser ausgibt.
		      %Wirkt wie Kinderkram nicht zum Funktionskapitel, möchte aber nochmal den Unterschied zwischen Ausgabe und Rückgabe dadurch nochmal klarer machen
		\item Kompiliert das angepasste Programm und lasst es im debugger Schritt für Schritt durchlaufen, setzt dafür wieder einen breakpoint für die \texttt{main}-Funktion.
		      Sobald der debugger euch anzeigt, als nächstes die Funktion ausführen zu wollen, \texttt{step} statt \texttt{next} aufrufen, sodass der debugger in die Funktion hineinspringt.
		\item Schreibt eine Funktion die ein \texttt{double} entgegen nimmt und das Quadrat davon zurück gibt.
		      (Hierbei sollt ihr keine Pakete wie \texttt{math.h} oder \texttt{cmath} benutzen.)
	\end{enumerate}
\end{praxis}

\begin{spiel}
	\begin{enumerate}
		\item Schreibt eine Funktion (nach der Funktion \texttt{collatz} und vor \texttt{main}), die einen \texttt{int} entgegen nimmt und die Anzahl der Schritte bestimmt bis die Folge bei der 1 angekommen ist und diese als \texttt{int} zurückgibt.
		      (Die Funktion sollte also die Signatur \cppinline{int schritte(int x)} haben.)
		      Probiert die Funktion aus.
		\item Versucht jetzt zwei Zahlen von der Nutzerin entgegen zu nehmen und vergleicht mithilfe von der gerade geschriebenen Funktion, welche Zahl mehr Schritte bis zur 1 braucht.
		\item Was passiert, wenn ihr in einer Funktion den \texttt{return}-Ausdruck vor dem Ende eurer Funktion benutzt?
		\item Vertauscht in \texttt{funktion.cpp} die Funktion \texttt{collatz} mit der Funktion \texttt{main} (verschiebt also die gesamte Funktion \texttt{collatz} an das Ende der Datei).
		      Versucht, die Datei zu kompilieren.
		      Was ist die Fehlermeldung des Compilers?
		\item Verschiebt die Funktion \texttt{collatz} \emph{in} die \texttt{main}-Funktion (also irgendwo nach der öffnenden geschweiften Klammern, aber vor die dazu gehörige schließende).
		      Versucht, die Datei zu kompilieren. Was ist die Fehlermeldung des Compilers?
		\item Implementiert die Funktion, die $x^n$ umsetzt, ignoriert dabei zunächst negative Exponenten. \\
		      (Wie in Praxis 3, sollt ihr auch hier keine vorgefertigten Pakete benutzen. \emph{Tipp:} Die Signatur ist bereits oben gegeben, für den Funktionsrumpf könnten sich Schleifen eignen.)
		\item Eure Funktion kann sich auch selbst aufrufen. Versucht damit eure Funktion auf negative Exponenten zu erweitern, indem ihr benutzt, dass gilt $x^{-n} = \Bigl(\frac{1.0}{x}\Bigr)^n$.
		\item Schaut euch eure bisherigen Lösungen an.
		      Findet ihr noch häufiger Stellen, an denen ihr einzelne Teilprogramme in Funktionen auslagern könnt?
	\end{enumerate}
\end{spiel}
