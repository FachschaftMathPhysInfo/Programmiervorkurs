\lesson{Die \Cpp Standardbibliothek}

Vielleicht habt ihr euch irgendwann gewundert, was eigentlich das
\texttt{std::} ist, was wir vor so viele Dinge schreiben. Warum müssen wir es
z.B. vor \texttt{string} schreiben, aber nicht vor \texttt{int}?

Die Antwort auf die Frage ist die \Cpp Standardbibliothek. So wie eigentlich
jede Programmiersprache, definiert sich \Cpp nicht nur durch die \emph{Syntax}
-- also die genaue Spezifikation, wie ein Quellcodeprogramm aufgebaut ist, wie
eine Anweisung aussieht und ob wir z.B. ein Semikolon am Ende jeder Anweisung
brauchen -- sondern auch über die im Sprachumfang enthaltene
Standardbibliothek, die einem nützliche Funktionen und Objekte für Ein- und
Ausgabe, komplexe Datentypen oder zur Interaktion mit dem Betriebssystem gibt.

\Cpp nutzt das Prinzip von so genannten \emph{Namespaces}. Das ist eine
Möglichkeit, eine Gruppe von Datentypen, Funktionen und Variablen unter einem
gemeinsamen Namen zu verpacken. Stellt euch vor, ihr wollt in eurem Programm
eine Funktion \texttt{random} definieren. Ihr hättet ganz schön große Problem,
denn der Compiler wüsste dann, wenn ihr \texttt{random} schreibt nicht, ob ihr
eure eigene Funktion meint, oder ob ihr die Standard-\Cpp Funktion meint.

Aus diesem Grund leben alle Funktionen und Objekte der \Cpp Standardbibliothek
im Namespace \texttt{std}. Um auf sie zuzugreifen, müsst ihr dem Compiler
sagen, aus welchen Namespace ihr sie haben wollt, dazu schreibt ihr eben den
Namen des Namespaces und zwei Doppelpunkte vor den Namen der Variablen (oder
Funktion), also ist \texttt{std::cout} „Die Variable \texttt{cout} aus dem
Namespace \texttt{std}“.

\inputcpp{namespaces.cpp}

\begin{praxis}
    \begin{enumerate}
        \item Was gibt dieses Programm aus, wenn man es kompiliert und ausführt?
              Überlegt es euch zuerst selbst, dann probiert es aus.
    \end{enumerate}

    Wenn ihr wissen wollt, was die Standardbibliothek alles so für euch bereit
    stellt, könnt ihr euch in der Referenz der Standardbibliothek unter

    \url{http://www.cplusplus.com/reference/}

    umschauen. Es ist nicht ganz einfach, zu wissen, wo man dort findet, was man
    sucht, in dem Fall kann Google ein im Regelfall ganz gut helfen. Wenn man
    einmal weiß, \emph{was} man sucht, findet man in der Referenz vor allem,
    \emph{wie} man es benutzt.

    Die Standardbibliothek ist aufgeteilt auf so genannt \emph{Headerdateien}, die
    wir mittels \texttt{\#include} benutzen können. Diese Header sind, worunter ihr
    zuerst wählt, wenn ihr auf obige url geht. Jeder Header definiert dann eine
    Menge an Funktionen, Typen und Klassen (was genau eine Klasse ist, lernt ihr
    spätestens in der Vorlesung).

    \begin{enumerate}[resume]
        \item Findet in der \Cpp-Referenz eine Funktion, um die aktuelle Zeit
              auszugeben. Schreibt ein Programm, welches die Aktuelle Zeit ausgibt
              (es reicht, einen so genannten \emph{Unix timestamp}\footnote{Der
                  Unix-Timestamp ist eine einzelne Zahl, die alle Sekunden seit dem
                  1.1.1970 anzeigt und die also jede Sekunde eins größer wird} auszugeben).
              Ihr könnt die Ausgabe eures Programms mit der Ausgabe von \texttt{date
                  +\%s} vergleichen, um es zu testen.
        \item Mit der Funktion \texttt{rand()} könnt ihr Zufallszahlen generieren
              (ihr braucht dazu den Header \texttt{<cstdlib>}). Schreibt ein
              Programm, welches vom Benutzer eine Zahl entgegennimmt und diese Anzahl
              Zufallszahlen ausgibt. Führt das Programm mehrfach aus. Was fällt auf?
        \item Konsultiert die \Cpp-Referenz, um heraus zu finden, wo das Problem
              liegt. Könnt ihr es beheben?
    \end{enumerate}
\end{praxis}

\textbf{Praxis:}
\begin{enumerate}
	\item Findet in der \Cpp-Referenz eine Funktion, um die aktuelle Zeit
		auszugeben. Schreibt ein Programm, welches die Aktuelle Zeit ausgibt
		(es reicht, einen so genannten \emph{Unix timestamp}\footnote{Der
		Unix-Timestamp ist eine einzelne Zahl, die alle Sekunden seit dem
	1.1.1970 anzeigt und die also jede Sekunde eins größer wird} auszugeben).
	Ihr könnt die Ausgabe eures Programms mit der Ausgabe von \texttt{date
		+\%s} vergleichen, um es zu testen.
    \item Mit der Funktion \texttt{rand()} könnt ihr Zufallszahlen generieren
        (ihr braucht dazu den Header \texttt{<cstdlib>}). Schreibt ein
        Programm, welches vom Benutzer eine Zahl entgegennimmt und diese Anzahl
        Zufallszahlen ausgibt. Führt das Programm mehrfach aus. Was fällt auf?
    \item Konsultiert die \Cpp-Referenz, um heraus zu finden, wo das Problem
        liegt. Könnt ihr es beheben?
\end{enumerate}

\newpage

\textbf{Quiz 14}\\
\textit{Welche Funktionen sind in der Standardbibliothek?}
\begin{enumerate}[label=\alph*)]
    \item \texttt{cout}
    \item \texttt{cin}
    \item \texttt{sqrt}
    \item \texttt{cerr}
\end{enumerate}
