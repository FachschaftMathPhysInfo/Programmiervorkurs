\chapter[Objektorientierung]{Einführung in Objektorientierung \& Klassen}
\setdirname{classes}
\pagestyle{empty}

Nachdem wir uns im ersten Kapitel die Basics kennen gelernt haben, wollen wir uns nun mit fortgeschrittenen Konzepten beschäftigen.
Das zentrale Thema dieses Kapitel, ist die \emph{Objektorientierung}.
Dabei geht es darum, zusammengehörige Daten und Funktionen zu bündeln.
Diese Bündel stellen dann eigene Datentypen dar, ein Beispiel für einen so definierte \glqq{}Klasse\grqq{} haben wir bereits im ersten Kapitel kennen gelernt, nämlich den \cppinline{std::vector}.

Um eigene Datentypen kennen zu lernen, wollen wir uns zunächst mit einfachen Varianten beschäftigen und diese in den Lektionen nach und nach komplexer machen.

\pagestyle{fancy}
\lesson{Enums \& Ansi}

Da die Basics der \Cpp-Programmierung nun verinnerlicht sind, fangen wir an uns eigene Datentypen zu definieren.
Eine der einfachsten Kategorien von Datentypen sind \emph{Enum}s. Der Hauptzweck von Enums ist das Zuordnen von Bedeutungen zu Nummerierungen.
Sie ermöglichen z.B. das Verwenden von leicht lesbaren Begriffen an Stellen, die eigentlich ganze Zahlen erfordern. \\
in folgendem Codebeispiel existiert eine Funktion \texttt{runden}, die abhängig der als \emph{mode} angegebenen Zahl entweder kaufmännnisch rundet, bzw. ab- oder aufrundet.

\inputcpp{runden.cpp}

Das ist relativ unschön und unintuitiv zu benutzen. Man weiß zum Beispiel nicht, wie viele modes es überhaupt gibt, wenn man den Code nicht genau ließt. Dies kann zu unnötigen Fehlern führen, wie man beim letzten Aufruf von \texttt{runden} sieht. Natürlich könnte man für den \emph{mode} auch den Typ String wählen, diese Verbrauchen aber sehr viel Platz und sind damit völlige Verschwendung für diese Aufgabe. Besser lässt sich das mit einem Enum lösen:

\inputcpp{runden2.cpp}

In den geschweiften Klammer wird angegeben, welche Zustände der Datentyp annehmen kann.
Intern werden diese von 0 beginnend durchnummeriert, was jedoch für viele Zwecke unwichtig ist.
Mit einem '='-Zeichen hinter dem Zustand kann diesem explizit eine Zahl zugeweisen werden, die dieser repräsentiert.
Dies kann in manchen Fällen sinnvoll sein, da Enums einfach in Integer verwandelt werden können (und anders herum).

\begin{praxis}
    \begin{enumerate}
        \item Erweitere \texttt{runden2.cpp} so, dass auch negative Eingaben richtig gerundet werden. Bei Abrunden sollte die Eingabe -1.4 also auf 2 und bei Aufrunden auf 1 gerundet werden.
        \item Füge zwei weitere modes hinzu, die jeweils zur Null und von der Null weg runden.
    \end{enumerate}
\end{praxis}
\lesson{Structs}
\registerfile{struct.cpp}

Wir sind bisher dazu in der Lage Variablen von vielen verschiedenen Typen anzulegen.
Wir wollen uns jetzt damit beschäftigen wie wir dreidimensionale Vektoren in \Cpp implementieren können.

Unsere bisherigen Datentypen sind dafür leider eher ungeeignet.
Wir könnten zwar einen Vektor simulieren indem wir drei \texttt{double} benutzen, allerdings haben wir bisher keine Möglichkeit, eine Funktion zu schreiben, die mehrere Zahlen auf einmal zurück gibt.

Um Datentypen zu bündeln, gibt es die sogenannten \emph{Structs}.
Diese besitzen verschiedene \emph{Attribute}, zum Beispiel würde ein Vektor drei verschiedene Attribute vom Typ \texttt{double} besitzen.
Das könnte zum Beispiel so aussehen:

\inputcpp{struct.cpp}

%Jetzt ist es immer noch umständlich einen neuen Vektor zu initialisieren, da man einzeln $x$, $y$ und $z$ festlegen muss.
%An dieser Stelle wäre eine Möglichkeit, diesen Teil in eine eigene Funktion auszulagern.
%Da das Initialisieren von Attributen eine häufige Aufgabe ist, gibt es sogenannte \emph{Konstruktoren}, die vordefinierte Parameter entgegen nehmen und damit einen bestimmten Struct initialisieren können.
%Das sieht dann zum Beispiel so aus:
%
%\includecpp{konst.cpp}

\begin{praxis}
    \begin{enumerate}
        \item Schreibt einen neuen Struct, der ein Datum folgender Art repräsentiert: \texttt{"1 Januar 1971"}.
              Dieser Struct sollte also drei Attribute haben:
              ein \texttt{int}, um das Jahr zu speichern,
              ein \texttt{std::string} für den Monat
              und ein weiterer \texttt{int} für den Tag.

        \item Erstellt eine Funktion, die einen \texttt{std::string} als Parameter entgegen nimmt, die Nutzerin mit der Ausgabe des Strings nach einem Datum fragt und dann das Datum mit eurem erstellten Struct zurück gibt.

        \item Fragt mit dieser Funktion die Benutzerin nach ihrem Geburtsdatum. Überprüft, ob ihr am gleichen Tag Geburtstag habt.
              Außerdem solltet ihr überprüfen, ob ihr im gleichen Jahr geboren seid.
    \end{enumerate}
\end{praxis}
\begin{praxis}[(Quadratsfunktion)]

    \begin{enumerate}
        \item Schreibt einen Struct \texttt{Point2D}, der zweidimensionale Punkte
              repräsentieren soll. ($a = (x, y) \in \mathbb{R}^2$)
        \item In einem früheren Kapitel habt ihr gelernt, wie man Arrays von
              Datentypen anlegen kann. Erstellt ein Array aus 100 \texttt{Point2D}s, bei dem die
              x-Werte von 0 bis 99 gehen.
        \item Berechnet dann für jeden \texttt{x}-Wert das entsprechende \texttt{y},
              indem ihr $y = x^2$ berechnet.
    \end{enumerate}
\end{praxis}

\begin{spiel}
    \begin{enumerate}
        \item ??
    \end{enumerate}
\end{spiel}

\lesson{Klassen}
\registerfile{class.cpp}

Eine komplexere Art von Datentypen als Enums sind die sogenannten Klassen. Diese lassen sich als eine Art Gegenstand vorstellen, der verschiedene Eigenschaften hat.

Wenn zum Beispiel eine Videothek ihre DVD-Sammlung verwalten möchte, wäre eine Möglichkeit, zu implementieren, jede DVD durch eine Klasse darzustellen. Dabei ist die Klasse dann ein Baupaln für die spären Datentypen, die erstell werden. Diese DVD-Klasse könnte dann Attribute für den Titel, ob sie zurzeit ausgeliehen ist, eine Möglichkeit, zu speichern, wann sie zurück gegeben werden muss, und einen Zähler, um zu speichern, wie oft die DVD schon ausgeliehen wird.

Nun wird es irgendwann passieren, dass jemand eine DVD ausleihen möchte. Dafür sollte es also eine leichte Möglichkeit geben, um den Ausleihstatus zu ändern und automatisch das Rückgabedatum auf einen sinnvollen Wert zu setzen.
Eine erfahrerne Benutzerin wird jetzt natürlich an Funktionen denken. Für die
Übersichtlichkeit des Programmes ist es sehr vorteilhaft, die Funktionen einer
Klasse zu bündeln. Dafür werden diese in der Klasse definiert und gehören so als
\emph{Memberfunktion} zur Klasse.


Eine Klasse kann Information haben, die für alle zugänglich sein sollten, und Informationen, die nicht oder nur eingeschränkt zugreifbar sein sollten. Zum Beispiel ist es sinnvoll, dass jede Benutzerin die Möglichkeit hat DVDs auszuleihen, deshalb wird die Methode ausleihen als \emph{public} definiert.
Da es aber auch klasseninterne Variablen und Funktionen gibt, wie hier zum
Beispiel die DVD-Sammlung selbst, ergibt es auch Sinn Teile der Klasse als
\emph{private} zu deklarieren. Wir modellieren also durch die Einteilung in public \&
private also Eigenschaften, die wir bisher nicht abbilden konnten. \\
Per Default ist alles in einer Klasse \emph{private}.

Jetzt gibt es auch noch weitere Attribute wie den Name der Sammlung, der nicht
so ganz in beide Kategorien passt. Da Benutzerinnen den Titel zwar auslesen müssen, aber den Titel nicht einfach verändern sollen. Dafür ist die übliche Vorgehensweise, \emph{getter}- bzw \emph{setter}-Methoden zu definieren. Eine getter-Methode sollte das entsprechende Attribut (zum Beispiel den Titel) zurückgeben. Wohingegen eine setter-Methode einen Parameter des Types des entsprechenden Attributes entgegen nimmt und dann den Wert intern ändert.

\inputcpp{class.cpp}


\pagestyle{empty}
