\chapter[Objektorientierung]{Einführung in Objektorientierung \& Klassen}
\setdirname{classes}
\pagestyle{empty}

Nachdem wir uns im ersten Kapitel die Basics kennen gelernt haben, wollen wir uns nun mit fortgeschrittenen Konzepten beschäftigen.
Das zentrale Thema dieses Kapitel, ist die \emph{Objektorientierung}.
Dabei geht es darum, zusammengehörige Daten und Funktionen zu bündeln.
Diese Bündel stellen dann eigene Datentypen dar, ein Beispiel für einen so definierte \glqq{}Klasse\grqq{} haben wir bereits im ersten Kapitel kennen gelernt, nämlich den \cppinline{std::vector}.

Um eigene Datentypen kennen zu lernen, wollen wir uns zunächst mit einfachen Varianten beschäftigen und diese in den Lektionen nach und nach komplexer machen.

\pagestyle{fancy}
\inputlektion{enum}
\inputlektion{struct}
\inputlektion{classes}

\pagestyle{empty}
