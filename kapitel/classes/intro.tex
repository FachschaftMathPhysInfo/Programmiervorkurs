\chapter[Objektorientierung]{Einführung in Objektorientierung \& Klassen}
\renewcommand{\filesource}{files/classes}
\pagestyle{empty}

Nachdem wir uns im ersten Kapitel die Basics kennen gelernt haben, wollen wir uns nun mit fortgeschrittenen Konzepten beschäftigen.
Das zentrale Thema dieses Kapitel, ist die \emph{Objektorientierung}.
Dabei geht es darum, zusammengehörige Daten und Funktionen zu bündeln.
Diese Bündel stellen dann eigene Datentypen dar, wir haben zum Beispiel schon im ersten Kapitel den \cppinline{std::vector} kennen gelernt.

\pagestyle{fancy}
\lesson{Enums}
\registerfile{runden.cpp}
\registerfile{runden2.cpp}

Wir wollen jetzt mit unserem ersten eigenen Datentyp anfangen.
Die einfachste Kategorie von Datentypen sind \emph{Enums}.
Um einem Enum zu definieren, geben wir einfach alle möglichen Werte explizit an.
Im Program können wir dann die möglichen Werte wie konstante Begriffe verwenden.
Zum Beispiel könnten wir einen Enum \enquote{Frucht} definieren, der aus den möglichen Werten \enquote{Apfel}, \enquote{Birne} und \enquote{Banane} besteht.
Wenn wir dann eine Variable mit unserem neuen Enum-Typen haben, ist dieser Variable einer dieser drei Werte zugeordnet.

Sinnvoll wird das natürlich erst, wenn wir damit einen bestimmten Zweck verfolgen.
In diesem Fall bietet es sich dafür an, in einem Programm verschiedene Fälle zu unterscheiden.
Ein Beispiel bei dem wir zwischen verschiedenen Fällen unterscheiden wollen, ist das Runden von reellen Zahlen.
Dort können wir beispielsweise kaufmännnisch Runden und Auf- oder Abrunden.
Mit unseren bisherigen Mitteln könnten wir das lösen, indem wir jeder Möglichkeit eine Nummer zuweisen und dann mit \cppinline{if} und \cppinline{else} die richtige Rechnung durchführen.
Das könnte dann beispielsweise so aussehen:

\inputcpp{runden.cpp}

Das ist eine sehr fehleranfällige Art, dieses Problem zu lösen.
Eine andere Programmiererin weiß zum Beispiel nicht -- ohne den Code genau anzuschauen -- welcher \cppinline{mode} was tut und wie viele überhaupt existieren.
Dies kann zu unnötigen Fehlern führen, wie man beim letzten Aufruf von \cppinline{runden} sieht.
Natürlich könnte man für den \emph{mode} auch den Typ \cppinline{std::string} wählen, diese verbrauchen aber sehr viel Platz und sind damit völlige Verschwendung für diese Aufgabe.
Besser lässt sich das mit einem Enum lösen:

\inputcpp{runden2.cpp}

In den geschweiften Klammer wird angegeben, welche Zustände der Datentyp annehmen kann.
Im restlichen Code können dann einfach die möglichen Zustände wie echte Worte benutzt werden.

Das Wort enum kommt vom englischen \enquote{enumerate} was so viel wie \enquote{aufzählen} bedeutet.
Und das ist auch was der Compiler intern macht:
Alle möglichen Werte von 0 beginnend durchnummeriert, wenn man einem bestimmten Zustand eine feste Zahl zuweisen will, kann man das mit einem \cppinline{=}-Zeichen machen.
Dafür schreibt man einfach \cppinline{kaufmaennisch=5} und schon ist \cppinline{kaufmaennisch} intern durch die 5 repräsentiert.
Das kann dafür benutzt werden, um Enums zu \cppinline{int}s umzuwandeln und andersrum, das funktioniert so
\begin{center}
    \cppinline{int x = (int) mode;} \qquad oder so \qquad \cppinline{Runden mode = (Runden) x;}   
\end{center}
Meistens ist es jedoch -- wie auch in unserem Beispiel -- unwichtig.

\begin{praxis}
    \begin{enumerate}
        \item Erweitere \texttt{runden2.cpp} so, dass auch negative Eingaben richtig gerundet werden. Bei Abrunden sollte die Eingabe -1.4 also auf 2 und bei Aufrunden auf 1 gerundet werden.
        \item Füge zwei weitere modes hinzu, die jeweils zur Null und von der Null weg runden.
    \end{enumerate}
\end{praxis}

\lesson{Struct und Konstruktor} % Anderer Name?

Wir sind bisher dazu in der Lage Variablen von vielen, verschiedenen Typen anzulegen.
Wir wollen uns jetzt damit beschäftigen wie wir dreidimensionale Vektoren in \Cpp implementieren können.

Unsere bisherigen Datentypen sind dafür leider eher ungeeignet.
Wir könnten zwar einen Vektor simulieren indem wir drei \texttt{double} benutzen, allerdings haben wir bisher keine Möglichkeit, eine Funktion zu schreiben, die mehrere Zahlen auf einmal zurück gibt.

Um Datentypen zu bündeln, gibt es die sogenannten \emph{Structs}.
Diese besitzen verschiedene \emph{Attribute}, zum Beispiel würde ein Vektor drei verschiedene Attribute vom Typ \texttt{double} besitzen.
Das könnte zum Beispiel so aussehen:

\inputcpp{../files/struct.cpp}

%Jetzt ist es immer noch umständlich einen neuen Vektor zu initialisieren, da man einzeln $x$, $y$ und $z$ festlegen muss.
%An dieser Stelle wäre eine Möglichkeit, diesen Teil in eine eigene Funktion auszulagern.
%Da das Initialisieren von Attributen eine häufige Aufgabe ist, gibt es sogenannte \emph{Konstruktoren}, die vordefinierte Parameter entgegen nehmen und damit einen bestimmten Struct initialisieren können.
%Das sieht dann zum Beispiel so aus:
%
%\includecpp{konst.cpp}

\textbf{Praxis:}
\begin{enumerate}
    \item Schreibt ein neues Struct, um ein Datum zu repräsentieren.
Dieser Struct sollte also drei Attribute haben:
    ein \texttt{int}, um das Jahr zu speichern,
    ein \texttt{std::string} für den Monat
    und ein weiterer \texttt{int} für den Tag.

    \item Erstellt eine Funktion, die einen \texttt{std::string} als Parameter entgegen nimmt, die Nutzerin mit der Ausgabe des Strings nach einem Datum fragt und dann das Datum mit eurem erstellten Struct zurück gibt.

    \item Fragt mit dieser Funktion die Benutzerin nach ihrem Geburtsdatum. Überprüft, ob ihr am gleichen Tag Geburtstag habt.
Außerdem solltet ihr überprüfen, ob ihr im gleichen Jahr geboren seid.
\end{enumerate}

\textbf{Spiel:}
\begin{enumerate}
    \item ??
\end{enumerate}

\include{kapitel/classes/operator}
\lesson{Klassen}

Eine komplexere Art von Datentypen als Enums sind die sogenannten Klassen. Diese lassen sich als eine Art Gegenstand vorstellen, der verschiedene Eigenschaften hat.

Wenn zum Beispiel eine Videothek ihre DVD-Sammlung verwalten möchte, wäre es eine Möglichkeit jede DVD durch eine Klasse darzustellen. Dabei ist die Klasse dann ein Bauplan für die späteren Datentypen, die erstellt werden. Diese DVD-Klasse könnte dann Attribute für den Titel, ob sie zurzeit ausgeliehen ist, eine Möglichkeit zu speichern, wann sie zurück gegeben werden muss, und einen Zähler, um zu speichern, wie oft die DVD schon ausgeliehen wird, enthalten.

Nun wird es irgendwann passieren, dass jemand eine DVD ausleihen möchte. Dafür sollte es also eine leichte Möglichkeit geben, um den Ausleihstatus zu ändern und automatisch das Rückgabedatum auf einen sinnvollen Wert zu setzen.
Eine erfahrerne Benutzerin wird jetzt natürlich an Funktionen denken. Für die
Übersichtlichkeit des Programmes ist es sehr vorteilhaft, die Funktionen einer
Klasse zu bündeln. Dafür werden diese in der Klasse definiert und gehören so als
\emph{Memberfunktion} zur Klasse.


Eine Klasse kann Information haben, die für alle zugänglich sein sollten, und Informationen, die nicht oder nur eingeschränkt zugreifbar sein sollten. Zum Beispiel ist es sinnvoll, dass jede Benutzerin die Möglichkeit hat DVDs auszuleihen, deshalb wird die Methode ausleihen als \emph{public} definiert.
Da es aber auch klasseninterne Variablen und Funktionen gibt, wie hier zum
Beispiel die DVD-Sammlung selbst, ergibt es auch Sinn Teile der Klasse als
\emph{private} zu deklarieren. Wir modellieren also durch die Einteilung in public \&
private also Eigenschaften, die wir bisher nicht abbilden konnten. \\
Per Default ist alles in einer Klasse \emph{private}.

Jetzt gibt es auch noch weitere Attribute wie den Name der Sammlung, der nicht
so ganz in beide Kategorien passt. Da Benutzerinnen den Titel zwar auslesen müssen, aber den Titel nicht einfach verändern sollen. Dafür ist die übliche Vorgehensweise, \emph{getter}- bzw \emph{setter}-Methoden zu definieren. Eine getter-Methode sollte das entsprechende Attribut (zum Beispiel den Titel) zurückgeben. Wohingegen eine setter-Methode einen Parameter des Types des entsprechenden Attributes entgegen nimmt und dann den Wert intern ändert.

\inputcpp{class.cpp}


\pagestyle{empty}
