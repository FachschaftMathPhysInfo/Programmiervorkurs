\lesson{Structs}
\registerfile{struct.cpp}

Wir sind bisher dazu in der Lage Variablen von vielen verschiedenen Typen anzulegen.
Wir wollen uns jetzt damit beschäftigen wie wir dreidimensionale Vektoren in \Cpp implementieren können.

Unsere bisherigen Datentypen sind dafür leider eher ungeeignet.
Wir könnten zwar einen Vektor simulieren indem wir drei \texttt{double} benutzen, allerdings haben wir bisher keine Möglichkeit, eine Funktion zu schreiben, die mehrere Zahlen auf einmal zurück gibt.

Um Datentypen zu bündeln, gibt es die sogenannten \emph{Structs}.
Diese besitzen verschiedene \emph{Attribute}, zum Beispiel würde ein Vektor drei verschiedene Attribute vom Typ \texttt{double} besitzen.
Das könnte zum Beispiel so aussehen:

\inputcpp{struct.cpp}

%Jetzt ist es immer noch umständlich einen neuen Vektor zu initialisieren, da man einzeln $x$, $y$ und $z$ festlegen muss.
%An dieser Stelle wäre eine Möglichkeit, diesen Teil in eine eigene Funktion auszulagern.
%Da das Initialisieren von Attributen eine häufige Aufgabe ist, gibt es sogenannte \emph{Konstruktoren}, die vordefinierte Parameter entgegen nehmen und damit einen bestimmten Struct initialisieren können.
%Das sieht dann zum Beispiel so aus:
%
%\includecpp{konst.cpp}

\begin{praxis}
    \begin{enumerate}
        \item Schreibt einen neuen Struct, der ein Datum folgender Art repräsentiert: \texttt{"1 Januar 1971"}.
              Dieser Struct sollte also drei Attribute haben:
              ein \texttt{int}, um das Jahr zu speichern,
              ein \texttt{std::string} für den Monat
              und ein weiterer \texttt{int} für den Tag.

        \item Erstellt eine Funktion, die einen \texttt{std::string} als Parameter entgegen nimmt, die Nutzerin mit der Ausgabe des Strings nach einem Datum fragt und dann das Datum mit eurem erstellten Struct zurück gibt.

        \item Fragt mit dieser Funktion die Benutzerin nach ihrem Geburtsdatum. Überprüft, ob ihr am gleichen Tag Geburtstag habt.
              Außerdem solltet ihr überprüfen, ob ihr im gleichen Jahr geboren seid.
    \end{enumerate}
\end{praxis}
\begin{praxis}[(Quadratsfunktion)]

    \begin{enumerate}
        \item Schreibt einen Struct \texttt{Point2D}, der zweidimensionale Punkte
              repräsentieren soll. ($a = (x, y) \in \mathbb{R}^2$)
        \item In einem früheren Kapitel habt ihr gelernt, wie man Arrays von
              Datentypen anlegen kann. Erstellt ein Array aus 100 \texttt{Point2D}s, bei dem die
              x-Werte von 0 bis 99 gehen.
        \item Berechnet dann für jeden \texttt{x}-Wert das entsprechende \texttt{y},
              indem ihr $y = x^2$ berechnet.
    \end{enumerate}
\end{praxis}

\begin{spiel}
    \begin{enumerate}
        \item ??
    \end{enumerate}
\end{spiel}
