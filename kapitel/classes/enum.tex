\lesson{Enums}
\registerfile{runden.cpp}
\registerfile{runden2.cpp}

Wir wollen jetzt mit unserem ersten eigenen Datentyp anfangen.
Die einfachste Kategorie von Datentypen sind \emph{Enums}.
Um einem Enum zu definieren, geben wir einfach alle möglichen Werte explizit an.
Im Program können wir dann die möglichen Werte wie konstante Begriffe verwenden.
Zum Beispiel könnten wir einen Enum \enquote{Frucht} definieren, der aus den möglichen Werten \enquote{Apfel}, \enquote{Birne} und \enquote{Banane} besteht.
Wenn wir dann eine Variable mit unserem neuen Enum-Typen haben, ist dieser Variable einer dieser drei Werte zugeordnet.

Sinnvoll wird das natürlich erst, wenn wir damit einen bestimmten Zweck verfolgen.
In diesem Fall bietet es sich dafür an, in einem Programm verschiedene Fälle zu unterscheiden.
Ein Beispiel bei dem wir zwischen verschiedenen Fällen unterscheiden wollen, ist das Runden von reellen Zahlen.
Dort können wir beispielsweise kaufmännnisch Runden und Auf- oder Abrunden.
Mit unseren bisherigen Mitteln könnten wir das lösen, indem wir jeder Möglichkeit eine Nummer zuweisen und dann mit \cppinline{if} und \cppinline{else} die richtige Rechnung durchführen.
Das könnte dann beispielsweise so aussehen:

\inputcpp{runden.cpp}

Das ist eine sehr fehleranfällige Art, dieses Problem zu lösen.
Eine andere Programmiererin weiß zum Beispiel nicht -- ohne den Code genau anzuschauen -- welcher \cppinline{mode} was tut und wie viele überhaupt existieren.
Dies kann zu unnötigen Fehlern führen, wie man beim letzten Aufruf von \cppinline{runden} sieht.
Natürlich könnte man für den \emph{mode} auch den Typ \cppinline{std::string} wählen, diese verbrauchen aber sehr viel Platz und sind damit völlige Verschwendung für diese Aufgabe.
Besser lässt sich das mit einem Enum lösen:

\inputcpp{runden2.cpp}

In den geschweiften Klammer wird angegeben, welche Zustände der Datentyp annehmen kann.
Im restlichen Code können dann einfach die möglichen Zustände wie echte Worte benutzt werden.

Das Wort enum kommt vom englischen \enquote{enumerate} was so viel wie \enquote{aufzählen} bedeutet.
Und das ist auch was der Compiler intern macht:
Alle möglichen Werte von 0 beginnend durchnummeriert, wenn man einem bestimmten Zustand eine feste Zahl zuweisen will, kann man das mit einem \cppinline{=}-Zeichen machen.
Dafür schreibt man einfach \cppinline{kaufmaennisch=5} und schon ist \cppinline{kaufmaennisch} intern durch die 5 repräsentiert.
Das kann dafür benutzt werden, um Enums zu \cppinline{int}s umzuwandeln und andersrum, das funktioniert so
\begin{center}
    \cppinline{int x = (int) mode;} \qquad oder so \qquad \cppinline{Runden mode = (Runden) x;}   
\end{center}
Meistens ist es jedoch -- wie auch in unserem Beispiel -- unwichtig.

\begin{praxis}
    \begin{enumerate}
        \item Erweitere \texttt{runden2.cpp} so, dass auch negative Eingaben richtig gerundet werden. Bei Abrunden sollte die Eingabe -1.4 also auf 2 und bei Aufrunden auf 1 gerundet werden.
        \item Füge zwei weitere modes hinzu, die jeweils zur Null und von der Null weg runden.
    \end{enumerate}
\end{praxis}
