\lesson{Input und Output}

Nachdem wir ein bisschen Vertrauen in die shell entwickelt haben und zumindest
bereits unser erstes Programm kompiliert, wollen wir nun etwas spannendere
Dinge tun. Nach wie vor müsst ihr nicht jede Zeile eures Programmes verstehen.
Sollte euch bei einer bestimmten Zeile trotzdem interessieren, was genau sie
tut, versucht doch eventuell sie zu entfernen, das Programm zu kompilieren und
schaut, was sich ändert.

Wir wollen uns nun mit grundlegendem input und output vertraut machen, denn
erst wenn euer Programm mit einer Benutzerin interagiert, wird es wirklich
nützlich. Wir haben in der ersten Lektion bereits \texttt{cout} (für
\emph{console out}) kennengelernt, um Dinge auszugeben. Nun nutzen wir
\texttt{cin}, um Eingaben des Benutzers entgegen zu nehmen. Jedes Programm
unter Linux (und übrigens auch Mac OS oder Windows) kann auf diese Weise
Eingaben von der Nutzerin entgegen nehmen und Ausgaben liefern. Das ist auch
der Grund, warum die Konsole so wichtig ist und es viele Dinge gibt, die nur
mittels einer Konsole gelöst werden können: Während es viele Stunden dauert,
ein grafisches Interface zu programmieren, über die man mit dem Programm mit
der Maus kommunizieren kann, kann praktisch jeder ein textbasiertes
Konsoleninterface schreiben. Linux ist ein Ökosystem mit einer gewaltigen
Anzahl tools für jeden denkbaren Zweck und bei den meisten haben die Autorinnen
sich nicht die Mühe gemacht, extra eine grafische Oberfläche zu entwickeln.

Nun aber direkt zur Praxis:

\begin{praxis}
    \begin{enumerate}
        \item Öffnet die Datei \texttt{vorkurs/basics/lektion03/helloyou.cpp} in eurem Texteditor
        \item Öffnet ein Terminal und wechselt in das Verzeichnis \texttt{vorkurs/basics/lektion03}
        \item Kompiliert im Terminal die Datei (\texttt{g++ -o helloyou
                  helloyou.cpp}) und führt sie aus (\texttt{./helloyou})
        \item Versucht verschiedene Eingaben an das Programm und beobachtet, was passiert
    \end{enumerate}

    \inputcpp{helloyou.cpp}
\end{praxis}

\begin{spiel}
    \begin{enumerate}
        \item Versucht, zu verstehen, was die einzelnen Teile des Programms tun. An
              welcher Stelle erfolgt die Eingabe? Was passiert dann damit?
        \item Erweitert das Programm um eigene Fragen und Ausgaben. Vergesst nicht,
              dass ihr das Programm nach jeder Änderung neu kompilieren und testen
              müsst.
    \end{enumerate}
\end{spiel}
