\lesson{Tic Tac Toe -- Teil 2}
% only few compilations are needed, write these by hand
\immediate\write\makefile{\mytab @g++ -c -o vorkurs/\dirname/lektion\twodigit{section}/frage_feld_nummer.o files/\dirname/ttt_closed/frage_feld_nummer.cpp}
\immediate\write\makefile{\mytab @g++ -c -o vorkurs/\dirname/lektion\twodigit{section}/gewinnerin.o files/\dirname/ttt_closed/gewinnerin.cpp}
\immediate\write\makefile{\mytab @g++ -c -o vorkurs/\dirname/lektion\twodigit{section}/gebe_feld_aus.o files/\dirname/ttt_closed/gebe_feld_aus.cpp}

Dies ist die letzte Lektion des ersten Kapitels. In der vorletzten Lektion
haben wir die grobe Struktur eines Tic Tac Toe Spiels implementiert, dafür
haben wir ein paar Funktionen benutzt, die uns gegeben waren. Nun wollen wir
diese Funktionen nachimplementieren („Implementieren“ heißt, dass man eine mehr
oder weniger formale Spezifikation in Programmcode umsetzt). Für eine
Beschreibung, was die Funktionen machen sollen, könnt ihr in der vorletzten
Lektion nachschauen. Damit ihr eure eigene Implementationen testen könnt, haben
wir noch einmal alle Funktionen mit beigefügt. Sie befinden sich in den Dateien
\texttt{frage\_feld\_nummer.o}, \texttt{gebe\_feld\_aus.o} und
\texttt{gewinnerin.o}.

Um eine Funktion zu implementieren, solltet ihr die dazugehörige
\texttt{extern}-Deklaration aus eurer \texttt{tictactoe.cpp} löschen und dann
die Funktion mit dem gleichen Namen (und den gleichen Parametern und
Rückgabetypen) selbst definieren und implementieren. Ihr könnt, wenn ihr z.B.
\texttt{gewinnerin} selbst implementiert habt, euer Programm mit

\texttt{g++ -o tictactoe tictactoe.cpp frage\_feld\_nummer.o gebe\_feld\_aus.o}

kompilieren. Je mehr Funktionen ihr selbst nachimplementiert, desto weniger
gegebene \texttt{.o}-files müsst ihr natürlich angeben.

Es gibt es dieses mal auch keine Nummern für die einzelnen Teile -- sucht euch
doch selbst aus, in welcher Reihenfolge ihr sie bearbeiten wollt, sie ist
ziemlich beliebig. Fangt am Besten mit dem Teil an, der euch am leichtesten
erscheint.

\begin{praxis}
    \begin{itemize}
        \item Implementiert \texttt{frage\_feld\_nummer} nach. Ihr solltet darauf
              achten, dass ihr in dieser Funktion auch testen müsst, ob ein gültiges
              Feld eingegeben wurde und ob das angegebene Feld leer ist.
        \item Implementiert \texttt{gebe\_feld\_aus} nach. Ihr könnt euch selbst
              aussuchen, wie ihr die Ausgabe gestalten wollt -- es muss nicht genauso
              aussehen, wie unser Vorschlag. Die
              Wikipedia\footnote{\url{http://en.wikipedia.org/wiki/Box-drawing_character}}
              kann euch z.B. helfen, ein schöneres Feld auszugeben. Fangt am Besten
              mit einer einfachen Ausgabe an und macht sie dann immer „fancier“.
        \item Implementiert \texttt{gewinnerin} nach. Bedenkt, dass ihr alle
              Möglichkeiten testet, auf die ein Gewinn möglich ist -- also 3
              Möglichkeiten, eine Reihe zu bilden, 3 Möglichkeiten, eine Spalte zu
              bilden und 2 Möglichkeiten für Diagonalen. Überlegt euch zunächst, wie
              ihr zwischen Feldnummer (0-8) und Reihen- bzw. Spaltennummer hin- und
              herrechnen könnt. Beachtet auch, dass es ein Unentschieden gibt, wenn
              alle Felder belegt sind, aber keine von beiden Spielerinnen gewonnen
              hat.
    \end{itemize}
\end{praxis}
