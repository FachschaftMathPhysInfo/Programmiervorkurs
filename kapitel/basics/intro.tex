\chapter{Die Basics}
\setdirname{basics}
\pagestyle{empty}

Im Ersten Kapitel werden wir die Grundlagen der Programmierung lernen.

Wir werden rausfinden, was ein Compiler ist, wie ein Programm abläuft und wie
man es startet. Wir werden Benutzereingaben verarbeiten und Ausgaben an die
Nutzerin geben. Wir lassen den Computer Rechnungen für uns anstellen und
lernen, was der Kontrollfluß ist -- und wie man ihn beeinflusst. Zuletzt werden
wir Vektoren kennenlernen und unser erstes nützliches Progamm schreiben.

\pagestyle{fancy}
\inputlektion{hello_world}
\inputlektion{konsole}
\inputlektion{input}
\inputlektion{fehler}
\inputlektion{variablen}
\inputlektion{manpages}
\inputlektion{arith}
\inputlektion{gdb}
\inputlektion{kontrollfluss}
\inputlektion{rechte}
\inputlektion{schleifen}
\inputlektion{style}
\inputlektion{funktionen}
\inputlektion{stdbib}
\inputlektion{vektoren}
\inputlektion{warning}
\inputlektion{tictactoe1}
\inputlektion{linker}
\inputlektion{tictactoe2}

\clearpage
\pagestyle{empty}

Ihr habt hiermit das erste Kapitel unseres
Programmiervorkurses abgeschlossen. Wir hoffen, ihr hattet dabei Spaß und habt
genug gelernt, um euch gut auf eure erste Programmiervorlesung vorbereitet zu
fühlen.

Rekapitulieren wir noch einmal unsere Eingangs formulierten „Lernziele“, was
wir uns wünschen würden, dass ihr aus diesem Vorkurs mitnehmt:
\begin{itemize}
    \item Ein Computer ist keine schwarze Magie
    \item Eine Konsole ist keine schwarze Magie
    \item Programmieren ist keine schwarze Magie
    \item Ihr wisst, wo ihr anfangt, wenn die Aufgabe ist „schreibt ein
        Programm, das\dots“
    \item Ihr entwickelt Spaß daran, Programmieraufgaben zu lösen
    \item Ihr wisst, was ihr tun könnt, wenn etwas nicht funktioniert
\end{itemize}

Von wie vielen davon habt ihr das Gefühl, sie erreicht zu haben? Wir würden uns
über euer Feedback freuen!
