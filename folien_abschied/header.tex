% !TEX root = vortrag.tex
% !TEX encoding = UTF-8 Unicode

\documentclass[t, ngerman]{beamer}

%% Pakete laden...
\usepackage[T1]{fontenc}
\usepackage[utf8]{inputenc}
\usepackage{
	amsmath,
	amsthm,
	amssymb,
	babel,
	bookmark,
	booktabs,
	graphicx,
	microtype,
	nicefrac,
	pifont,
	pgfpages,
	tikz,
}
\usepackage[overlay, absolute]{textpos}
\setlength{\TPHorizModule}{1mm}
\setlength{\TPVertModule}{1mm}


%% Design festlegen...
\mode<presentation>{
	%      \useoutertheme[subsection=false]{smoothbars}
	\useinnertheme{rectangles} % rectangles, circles, rounded
	\usecolortheme[RGB={153,0,0}]{structure}
	\definecolor{unihd}{RGB}{153,0,0}
	\definecolor{dark}{RGB}{115,0,0}
	\definecolor{light}{RGB}{241,229,229}
	\usecolortheme{whale}
	\usecolortheme{orchid}
	%	   \usecolortheme{beaver}
	%     \setbeamercovered{transparent}
	\beamertemplatenavigationsymbolsempty
	%      \setbeameroption{show notes on second screen}
	\setbeamertemplate{note page}[plain]
	\logo{\includegraphics[width=3.5cm]{fs-logo-small}}
}

%% nützliche Definitionen...
\graphicspath{{media/}}

%% Titelinformationen...
\title[Programmiervorkurs]{Abschlussworte}
\author[
	koebi
]{
	Jakob Schnell, Friedrich Schwedler und Patrick Dammann
}

\date{\vspace*{-2em}\\ \today}

\hypersetup{
	pdfauthor={Jakob Schnell},
	pdftitle={Programmiervorkurs},
	pdfsubject={hihi},
	pdfkeywords={1},
	pdfpagelayout={SinglePage},
}

%% Nützliche Commands...
\newcommand{\Ra}{\Rightarrow}
\newcommand{\R}{\mathbb{R}}
\newcommand{\Enc}{\mathrm{Enc}}
