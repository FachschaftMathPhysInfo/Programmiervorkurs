\lesson{Enums \& Ansi}

Da die Basics der \Cpp-Programmierung nur verinnerlicht sind, fangen wir an uns eigene Datentypen zu definieren.
Eine der einfachsten Kategorien von Datentypen sind \emph{Enum}s. Der Hauptzweck von Enums ist das Zuordnen von Bedeutungen zu Nummerierungen.
Sie ermöglichen z.B. das Verwenden von leicht lesbaren Begriffen an Stellen, die eigentlich ganze Zahlen erfordern. \\
in folgendem Codebeispiel existiert eine Funktion \emph{runden}, die abhängig vom angegebenen \emph{mode} entweder mathematisch rundet oder ab- bzw. aufrundet.

\inputcpp{runden.cpp}

Das ist relativ unschön und unintuitiv zu benutzen. Natürlich könnte man für den \emph{mode} auch den Typ String wählen,
diese Verbrauchen aber sehr viel Platz und sind damit völlige Verschwendung für diese Aufgabe.
Besser lässt sich das mit einem Enum lösen:

\inputcpp{runden2.cpp}

In den geschweiften Klammer wird angegeben, welche Zustände der Datentyp annehmen kann.
Intern werden diese von 0 beginnend durchnummeriert, was jedoch für viele Zwecke unwichtig ist.
Mit einem =-Zeichen hinter dem Zustand kann diesem explizit eine Zahl zugeweisen werden, die dieser repräsentiert.
Dies kann in manchen Fällen sinnvoll sein, da Enums einfach in Integer verwandelt werden können.

\textbf{Praxis:}
