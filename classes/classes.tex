\lesson{Klassen}

Eine komplexere Art von Datentypen als Enums sind die sogenannten Klassen. Diese lassen sich als eine Art Gegenstand vorstellen, der verschiedene Eigenschaften hat.

Wenn zum Beispiel eine Videothek ihre DVD-Sammlung verwalten möchte, wäre es eine Möglichkeit jede DVD durch eine Klasse darzustellen. Dabei ist die Klasse dann ein Bauplan für die späteren Datentypen, die erstellt werden. Diese DVD-Klasse könnte dann Attribute für den Titel, ob sie zurzeit ausgeliehen ist, eine Möglichkeit zu speichern, wann sie zurück gegeben werden muss, und einen Zähler, um zu speichern, wie oft die DVD schon ausgeliehen wird, enthalten.

Nun wird es irgendwann passieren, dass jemand eine DVD ausleihen möchte. Dafür sollte es also eine leichte Möglichkeit geben, um den Ausleihstatus zu ändern und automatisch das Rückgabedatum auf einen sinnvollen Wert zu setzen.
Eine erfahrerne Benutzerin wird jetzt natürlich an Funktionen denken. Für die
Übersichtlichkeit des Programmes ist es sehr vorteilhaft, die Funktionen einer
Klasse zu bündeln. Dafür werden diese in der Klasse definiert und gehören so als
\emph{Memberfunktion} zur Klasse.


Eine Klasse kann Information haben, die für alle zugänglich sein sollten, und Informationen, die nicht oder nur eingeschränkt zugreifbar sein sollten. Zum Beispiel ist es sinnvoll, dass jede Benutzerin die Möglichkeit hat DVDs auszuleihen, deshalb wird die Methode ausleihen als \emph{public} definiert.
Da es aber auch klasseninterne Variablen und Funktionen gibt, wie hier zum
Beispiel die DVD-Sammlung selbst, ergibt es auch Sinn Teile der Klasse als
\emph{private} zu deklarieren. Wir modellieren also durch die Einteilung in public \&
private also Eigenschaften, die wir bisher nicht abbilden konnten. \\
Per Default ist alles in einer Klasse \emph{private}.

Jetzt gibt es auch noch weitere Attribute wie den Name der Sammlung, der nicht
so ganz in beide Kategorien passt. Da Benutzerinnen den Titel zwar auslesen müssen, aber den Titel nicht einfach verändern sollen. Dafür ist die übliche Vorgehensweise, \emph{getter}- bzw \emph{setter}-Methoden zu definieren. Eine getter-Methode sollte das entsprechende Attribut (zum Beispiel den Titel) zurückgeben. Wohingegen eine setter-Methode einen Parameter des Types des entsprechenden Attributes entgegen nimmt und dann den Wert intern ändert.

\inputcpp{class.cpp}
